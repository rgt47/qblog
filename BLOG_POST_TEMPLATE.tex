% Options for packages loaded elsewhere
% Options for packages loaded elsewhere
\PassOptionsToPackage{unicode}{hyperref}
\PassOptionsToPackage{hyphens}{url}
\PassOptionsToPackage{dvipsnames,svgnames,x11names}{xcolor}
%
\documentclass[
  11pt,
]{article}
\usepackage{xcolor}
\usepackage[margin=0.75in]{geometry}
\usepackage{amsmath,amssymb}
\setcounter{secnumdepth}{-\maxdimen} % remove section numbering
\usepackage{iftex}
\ifPDFTeX
  \usepackage[T1]{fontenc}
  \usepackage[utf8]{inputenc}
  \usepackage{textcomp} % provide euro and other symbols
\else % if luatex or xetex
  \usepackage{unicode-math} % this also loads fontspec
  \defaultfontfeatures{Scale=MatchLowercase}
  \defaultfontfeatures[\rmfamily]{Ligatures=TeX,Scale=1}
\fi
\usepackage{lmodern}
\ifPDFTeX\else
  % xetex/luatex font selection
\fi
% Use upquote if available, for straight quotes in verbatim environments
\IfFileExists{upquote.sty}{\usepackage{upquote}}{}
\IfFileExists{microtype.sty}{% use microtype if available
  \usepackage[]{microtype}
  \UseMicrotypeSet[protrusion]{basicmath} % disable protrusion for tt fonts
}{}
\makeatletter
\@ifundefined{KOMAClassName}{% if non-KOMA class
  \IfFileExists{parskip.sty}{%
    \usepackage{parskip}
  }{% else
    \setlength{\parindent}{0pt}
    \setlength{\parskip}{6pt plus 2pt minus 1pt}}
}{% if KOMA class
  \KOMAoptions{parskip=half}}
\makeatother
% Make \paragraph and \subparagraph free-standing
\makeatletter
\ifx\paragraph\undefined\else
  \let\oldparagraph\paragraph
  \renewcommand{\paragraph}{
    \@ifstar
      \xxxParagraphStar
      \xxxParagraphNoStar
  }
  \newcommand{\xxxParagraphStar}[1]{\oldparagraph*{#1}\mbox{}}
  \newcommand{\xxxParagraphNoStar}[1]{\oldparagraph{#1}\mbox{}}
\fi
\ifx\subparagraph\undefined\else
  \let\oldsubparagraph\subparagraph
  \renewcommand{\subparagraph}{
    \@ifstar
      \xxxSubParagraphStar
      \xxxSubParagraphNoStar
  }
  \newcommand{\xxxSubParagraphStar}[1]{\oldsubparagraph*{#1}\mbox{}}
  \newcommand{\xxxSubParagraphNoStar}[1]{\oldsubparagraph{#1}\mbox{}}
\fi
\makeatother

\usepackage{color}
\usepackage{fancyvrb}
\newcommand{\VerbBar}{|}
\newcommand{\VERB}{\Verb[commandchars=\\\{\}]}
\DefineVerbatimEnvironment{Highlighting}{Verbatim}{commandchars=\\\{\}}
% Add ',fontsize=\small' for more characters per line
\usepackage{framed}
\definecolor{shadecolor}{RGB}{241,243,245}
\newenvironment{Shaded}{\begin{snugshade}}{\end{snugshade}}
\newcommand{\AlertTok}[1]{\textcolor[rgb]{0.68,0.00,0.00}{#1}}
\newcommand{\AnnotationTok}[1]{\textcolor[rgb]{0.37,0.37,0.37}{#1}}
\newcommand{\AttributeTok}[1]{\textcolor[rgb]{0.40,0.45,0.13}{#1}}
\newcommand{\BaseNTok}[1]{\textcolor[rgb]{0.68,0.00,0.00}{#1}}
\newcommand{\BuiltInTok}[1]{\textcolor[rgb]{0.00,0.23,0.31}{#1}}
\newcommand{\CharTok}[1]{\textcolor[rgb]{0.13,0.47,0.30}{#1}}
\newcommand{\CommentTok}[1]{\textcolor[rgb]{0.37,0.37,0.37}{#1}}
\newcommand{\CommentVarTok}[1]{\textcolor[rgb]{0.37,0.37,0.37}{\textit{#1}}}
\newcommand{\ConstantTok}[1]{\textcolor[rgb]{0.56,0.35,0.01}{#1}}
\newcommand{\ControlFlowTok}[1]{\textcolor[rgb]{0.00,0.23,0.31}{\textbf{#1}}}
\newcommand{\DataTypeTok}[1]{\textcolor[rgb]{0.68,0.00,0.00}{#1}}
\newcommand{\DecValTok}[1]{\textcolor[rgb]{0.68,0.00,0.00}{#1}}
\newcommand{\DocumentationTok}[1]{\textcolor[rgb]{0.37,0.37,0.37}{\textit{#1}}}
\newcommand{\ErrorTok}[1]{\textcolor[rgb]{0.68,0.00,0.00}{#1}}
\newcommand{\ExtensionTok}[1]{\textcolor[rgb]{0.00,0.23,0.31}{#1}}
\newcommand{\FloatTok}[1]{\textcolor[rgb]{0.68,0.00,0.00}{#1}}
\newcommand{\FunctionTok}[1]{\textcolor[rgb]{0.28,0.35,0.67}{#1}}
\newcommand{\ImportTok}[1]{\textcolor[rgb]{0.00,0.46,0.62}{#1}}
\newcommand{\InformationTok}[1]{\textcolor[rgb]{0.37,0.37,0.37}{#1}}
\newcommand{\KeywordTok}[1]{\textcolor[rgb]{0.00,0.23,0.31}{\textbf{#1}}}
\newcommand{\NormalTok}[1]{\textcolor[rgb]{0.00,0.23,0.31}{#1}}
\newcommand{\OperatorTok}[1]{\textcolor[rgb]{0.37,0.37,0.37}{#1}}
\newcommand{\OtherTok}[1]{\textcolor[rgb]{0.00,0.23,0.31}{#1}}
\newcommand{\PreprocessorTok}[1]{\textcolor[rgb]{0.68,0.00,0.00}{#1}}
\newcommand{\RegionMarkerTok}[1]{\textcolor[rgb]{0.00,0.23,0.31}{#1}}
\newcommand{\SpecialCharTok}[1]{\textcolor[rgb]{0.37,0.37,0.37}{#1}}
\newcommand{\SpecialStringTok}[1]{\textcolor[rgb]{0.13,0.47,0.30}{#1}}
\newcommand{\StringTok}[1]{\textcolor[rgb]{0.13,0.47,0.30}{#1}}
\newcommand{\VariableTok}[1]{\textcolor[rgb]{0.07,0.07,0.07}{#1}}
\newcommand{\VerbatimStringTok}[1]{\textcolor[rgb]{0.13,0.47,0.30}{#1}}
\newcommand{\WarningTok}[1]{\textcolor[rgb]{0.37,0.37,0.37}{\textit{#1}}}

\usepackage{longtable,booktabs,array}
\usepackage{calc} % for calculating minipage widths
% Correct order of tables after \paragraph or \subparagraph
\usepackage{etoolbox}
\makeatletter
\patchcmd\longtable{\par}{\if@noskipsec\mbox{}\fi\par}{}{}
\makeatother
% Allow footnotes in longtable head/foot
\IfFileExists{footnotehyper.sty}{\usepackage{footnotehyper}}{\usepackage{footnote}}
\makesavenoteenv{longtable}
\usepackage{graphicx}
\makeatletter
\newsavebox\pandoc@box
\newcommand*\pandocbounded[1]{% scales image to fit in text height/width
  \sbox\pandoc@box{#1}%
  \Gscale@div\@tempa{\textheight}{\dimexpr\ht\pandoc@box+\dp\pandoc@box\relax}%
  \Gscale@div\@tempb{\linewidth}{\wd\pandoc@box}%
  \ifdim\@tempb\p@<\@tempa\p@\let\@tempa\@tempb\fi% select the smaller of both
  \ifdim\@tempa\p@<\p@\scalebox{\@tempa}{\usebox\pandoc@box}%
  \else\usebox{\pandoc@box}%
  \fi%
}
% Set default figure placement to htbp
\def\fps@figure{htbp}
\makeatother





\setlength{\emergencystretch}{3em} % prevent overfull lines

\providecommand{\tightlist}{%
  \setlength{\itemsep}{0pt}\setlength{\parskip}{0pt}}



 
\usepackage[]{natbib}
\bibliographystyle{plainnat}


\renewcommand{\tableofcontents}{}
\usepackage{fvextra}
\DefineVerbatimEnvironment{Highlighting}{Verbatim}{breaklines,commandchars=\\\{\}}
\usepackage{fontspec}
\setmonofont{DejaVu Sans Mono}[Scale=0.9]
\usepackage{url}
\urlstyle{same}
\def\UrlBreaks{\do\/\do\-\do\_\do\.\do\:\do\?\do\&\do\=}
\makeatletter
\@ifpackageloaded{caption}{}{\usepackage{caption}}
\AtBeginDocument{%
\ifdefined\contentsname
  \renewcommand*\contentsname{Table of contents}
\else
  \newcommand\contentsname{Table of contents}
\fi
\ifdefined\listfigurename
  \renewcommand*\listfigurename{List of Figures}
\else
  \newcommand\listfigurename{List of Figures}
\fi
\ifdefined\listtablename
  \renewcommand*\listtablename{List of Tables}
\else
  \newcommand\listtablename{List of Tables}
\fi
\ifdefined\figurename
  \renewcommand*\figurename{Figure}
\else
  \newcommand\figurename{Figure}
\fi
\ifdefined\tablename
  \renewcommand*\tablename{Table}
\else
  \newcommand\tablename{Table}
\fi
}
\@ifpackageloaded{float}{}{\usepackage{float}}
\floatstyle{ruled}
\@ifundefined{c@chapter}{\newfloat{codelisting}{h}{lop}}{\newfloat{codelisting}{h}{lop}[chapter]}
\floatname{codelisting}{Listing}
\newcommand*\listoflistings{\listof{codelisting}{List of Listings}}
\makeatother
\makeatletter
\makeatother
\makeatletter
\@ifpackageloaded{caption}{}{\usepackage{caption}}
\@ifpackageloaded{subcaption}{}{\usepackage{subcaption}}
\makeatother
\usepackage{bookmark}
\IfFileExists{xurl.sty}{\usepackage{xurl}}{} % add URL line breaks if available
\urlstyle{same}
\hypersetup{
  pdftitle={Your Engaging Title Here},
  pdfauthor={Your Name},
  colorlinks=true,
  linkcolor={blue},
  filecolor={Maroon},
  citecolor={Blue},
  urlcolor={Blue},
  pdfcreator={LaTeX via pandoc}}


\title{Your Engaging Title Here}
\usepackage{etoolbox}
\makeatletter
\providecommand{\subtitle}[1]{% add subtitle to \maketitle
  \apptocmd{\@title}{\par {\large #1 \par}}{}{}
}
\makeatother
\subtitle{A compelling subtitle that expands on the main title}
\author{Your Name}
\date{2025-01-01}
\begin{document}
\maketitle


\begin{figure}[H]

{\centering \pandocbounded{\includegraphics[keepaspectratio]{../../images/posts/ucsd-geisel-library.jpg}}

}

\caption{UCSD Geisel Library - A hub for research and academic
discovery}

\end{figure}%

\emph{The Geisel Library at UC San Diego, where research and innovation
converge. \textbf{TEMPLATE NOTE}: This is your HERO IMAGE - the first
visual impression. Replace with an engaging, high-quality image relevant
to your topic. Consider: data visualizations, process diagrams,
screenshots, or professional stock photos. This image sets the tone and
draws readers into your content.}

\section{Introduction}\label{introduction}

In this post, we'll explore {[}topic/technique/problem{]}. This is
particularly relevant for {[}target audience{]} because
{[}motivation/problem statement{]}. As {[}Expert Name{]} notes in their
influential work, {[}brief quote or paraphrase that establishes
credibility and context{]} \citep{citation_key}.

By the end of this post, you'll be able to:

\begin{itemize}
\tightlist
\item
  {[}Learning objective 1{]}
\item
  {[}Learning objective 2{]}
\item
  {[}Learning objective 3{]}
\end{itemize}

\footnote{\textbf{Expert Insight:} ``{[}Compelling quote from industry
  expert that relates to your topic and adds credibility{]}''
  \emph{---{[}Expert Name{]}, {[}Title/Organization{]}}}

\section{Prerequisites and Setup}\label{prerequisites-and-setup}

Before we begin, ensure you have the following:

\textbf{Required Packages:}

\begin{Shaded}
\begin{Highlighting}[]
\CommentTok{\# Install required packages if not already installed}
\FunctionTok{install.packages}\NormalTok{(}\FunctionTok{c}\NormalTok{(}\StringTok{"package1"}\NormalTok{, }\StringTok{"package2"}\NormalTok{, }\StringTok{"package3"}\NormalTok{))}
\end{Highlighting}
\end{Shaded}

\textbf{Load Libraries:}

\begin{Shaded}
\begin{Highlighting}[]
\CommentTok{\# Replace with your actual packages}
\CommentTok{\# library(dplyr)}
\CommentTok{\# library(ggplot2)}
\CommentTok{\# library(readr)}
\end{Highlighting}
\end{Shaded}

\textbf{Sample Data:}

\begin{Shaded}
\begin{Highlighting}[]
\CommentTok{\# Replace with your actual data loading}
\CommentTok{\# data \textless{}{-} read\_csv("your\_data.csv")}
\CommentTok{\# data \textless{}{-} mtcars  \# Example with built{-}in data}
\end{Highlighting}
\end{Shaded}

\section{Main Section 1: {[}Descriptive
Heading{]}}\label{main-section-1-descriptive-heading}

{[}Explanation of first main concept{]}

\textbf{TEMPLATE NOTE:} Consider adding an in-text citation here to
support your explanation: ``Research has shown that
{[}finding/technique{]} significantly improves {[}outcome{]}
\citep{research_citation}.'' This adds academic rigor and credibility to
your technical content.

\begin{Shaded}
\begin{Highlighting}[]
\CommentTok{\# Replace with your actual example code}
\CommentTok{\# result \textless{}{-} your\_function(data)}
\CommentTok{\# print(result)}
\end{Highlighting}
\end{Shaded}

\subsection{Subsection 1.1: {[}More Specific
Topic{]}}\label{subsection-1.1-more-specific-topic}

{[}More detailed explanation or variation{]}

\footnote{\textbf{Pro Tip:} ``{[}Short, actionable tip or insight from a
  practitioner in your field{]}'' \emph{---{[}Practitioner Name{]},
  {[}Title/Company{]}}}

\begin{figure}[H]

{\centering \pandocbounded{\includegraphics[keepaspectratio]{../../images/posts/git.png}}

}

\caption{Optional supporting visualization with descriptive caption}

\end{figure}%

\emph{Example of {[}process/concept{]} in action. \textbf{TEMPLATE
NOTE}: This is an optional SUPPORTING IMAGE. Use sparingly - only if it
significantly enhances understanding. Good candidates: workflow
diagrams, before/after examples, or technical illustrations that clarify
complex concepts. Don't include images just for decoration.}

\section{Main Section 2:
{[}Implementation/Analysis{]}}\label{main-section-2-implementationanalysis}

{[}Detailed implementation or analysis{]}

\begin{Shaded}
\begin{Highlighting}[]
\CommentTok{\# Replace with your actual advanced example}
\CommentTok{\# advanced\_result \textless{}{-} complex\_analysis(data)}
\CommentTok{\# summary(advanced\_result)}
\end{Highlighting}
\end{Shaded}

\subsection{Subsection 2.1: {[}Handling Edge
Cases{]}}\label{subsection-2.1-handling-edge-cases}

{[}Discussion of potential issues and solutions{]}

\begin{Shaded}
\begin{Highlighting}[]
\CommentTok{\# Replace with your actual error handling code}
\CommentTok{\# tryCatch(\{}
\CommentTok{\#   risky\_operation(data)}
\CommentTok{\# \}, error = function(e) \{}
\CommentTok{\#   message("Error handled: ", e$message)}
\CommentTok{\# \})}
\end{Highlighting}
\end{Shaded}

\section{Main Section 3: {[}Results/Advanced
Applications{]}}\label{main-section-3-resultsadvanced-applications}

{[}Analysis of results or advanced applications{]}

\textbf{TEMPLATE NOTE:} This is an excellent location for another expert
quote or important security/best practice callout:

\footnote{\textbf{Security/Best Practice Alert:} ``{[}Important warning,
  best practice, or methodological consideration from authoritative
  source{]}'' \emph{---{[}Authority Name{]}, {[}Organization{]}}}

\begin{Shaded}
\begin{Highlighting}[]
\CommentTok{\# Replace with your actual final analysis}
\CommentTok{\# final\_plot \textless{}{-} ggplot(data, aes(x, y)) + }
\CommentTok{\#   geom\_point() +}
\CommentTok{\#   theme\_minimal()}
\CommentTok{\# print(final\_plot)}
\end{Highlighting}
\end{Shaded}

\begin{figure}[H]

{\centering \pandocbounded{\includegraphics[keepaspectratio]{../../images/posts/quarto.jpg}}

}

\caption{Data visualization workflow - from raw data to insights}

\end{figure}%

\emph{Just as Quarto streamlines document creation, {[}your
process/method{]} streamlines {[}relevant workflow{]}. \textbf{TEMPLATE
NOTE}: This is your MID-CONTENT ENGAGEMENT image (positioned
\textasciitilde50-60\% through the post). This creates crucial visual
rhythm that prevents reader fatigue in technical content. Replace with
your most important visualization: key results chart, process diagram,
screenshot of implementation, or compelling technical graphic that
reinforces your main points.}

\section{Main Section 4:
{[}Performance/Comparison{]}}\label{main-section-4-performancecomparison}

{[}Performance analysis or comparison with alternative approaches{]}

\textbf{Comparative Analysis:} Research by {[}Author{]} demonstrates
that {[}your approach{]} outperforms {[}alternative method{]} by
{[}quantified improvement{]} \citep{performance_citation}. This finding
is particularly relevant when {[}specific conditions or use cases{]}.

\begin{Shaded}
\begin{Highlighting}[]
\CommentTok{\# Replace with your actual benchmarking code}
\CommentTok{\# system.time(method1(data))}
\CommentTok{\# system.time(method2(data))}
\end{Highlighting}
\end{Shaded}

\section{Results and Key Findings}\label{results-and-key-findings}

Our analysis revealed several key findings:

\begin{enumerate}
\def\labelenumi{\arabic{enumi}.}
\tightlist
\item
  \textbf{{[}Key finding 1{]}}: {[}Brief explanation with numbers if
  applicable{]}
\item
  \textbf{{[}Key finding 2{]}}: {[}Brief explanation{]}
\item
  \textbf{{[}Key finding 3{]}}: {[}Brief explanation{]}
\end{enumerate}

\begin{figure}[H]

{\centering \pandocbounded{\includegraphics[keepaspectratio]{../../images/posts/ucsd-geisel-library.jpg}}

}

\caption{Research insights emerging from systematic analysis - like
knowledge discovered in UC San Diego's academic environment}

\end{figure}%

\emph{Just as the Geisel Library serves as a foundation for discovery,
our analysis provides a solid foundation for understanding {[}your
topic{]}. \textbf{TEMPLATE NOTE}: This is your RESULTS IMAGE - the
visual climax of your post. Replace with your most compelling findings
visualization: results chart, before/after comparison, summary
infographic, or key insights graphic. This image should visually
encapsulate your main conclusions and provide a satisfying visual payoff
for readers who've followed your technical journey.}

\section{Limitations and
Considerations}\label{limitations-and-considerations}

While this approach is effective, there are some important
considerations:

\subsection{Model Assumptions}\label{model-assumptions}

\begin{itemize}
\tightlist
\item
  \textbf{{[}Assumption 1{]}}: {[}e.g., Linearity assumption - check
  with residual plots{]}
\item
  \textbf{{[}Assumption 2{]}}: {[}e.g., Independence of observations{]}
\item
  \textbf{{[}Assumption 3{]}}: {[}e.g., Homoscedasticity - constant
  variance{]}
\end{itemize}

\subsection{Data Limitations}\label{data-limitations}

\begin{itemize}
\tightlist
\item
  \textbf{Sample size}: {[}Discussion of adequacy for conclusions{]}
\item
  \textbf{Generalizability}: {[}Population this applies to vs.~broader
  populations{]}
\item
  \textbf{Missing data}: {[}How missing values were handled and
  potential bias{]}
\end{itemize}

\subsection{Method Limitations}\label{method-limitations}

\begin{itemize}
\tightlist
\item
  \textbf{{[}Limitation 1{]}}: {[}Explanation and potential
  workarounds{]}
\item
  \textbf{{[}Limitation 2{]}}: {[}When this approach may not be
  appropriate{]}
\item
  \textbf{Performance considerations}: {[}Computational requirements,
  scalability{]}
\end{itemize}

\subsection{Security and Best Practice
Considerations}\label{security-and-best-practice-considerations}

\textbf{TEMPLATE NOTE:} Based on your domain, adapt these security
considerations:

\begin{itemize}
\tightlist
\item
  \textbf{Data Privacy}: {[}How sensitive data is handled,
  anonymization, compliance requirements{]}
\item
  \textbf{Code Security}: {[}Input validation, SQL injection prevention,
  secure coding practices{]}
\item
  \textbf{Reproducibility}: {[}Version control, dependency management,
  environment documentation{]}
\item
  \textbf{Ethical Considerations}: {[}Bias assessment, fairness,
  transparency requirements{]}
\item
  \textbf{Production Deployment}: {[}Security reviews, testing
  requirements, monitoring needs{]}
\end{itemize}

\textbf{Example Security Callout:} \footnote{\textbf{Security Alert:}
  ``Never store credentials in code repositories. Use environment
  variables or secure credential management systems.'' \emph{---Security
  Best Practices}}

\section{Future Extensions}\label{future-extensions}

This work could be extended in several directions:

\begin{itemize}
\tightlist
\item
  {[}Extension idea 1{]}
\item
  {[}Extension idea 2{]}
\item
  {[}Extension idea 3{]}
\end{itemize}

\section{Conclusion}\label{conclusion}

In this post, we've demonstrated {[}brief summary of what was
accomplished{]}. The key advantages of this approach are {[}main
benefits{]}.

\textbf{Next Steps:} - Try this technique with your own data -
Experiment with different parameters - Explore the additional resources
below

I encourage you to adapt this approach to your specific use case and
share your experiences in the comments below.

\footnote{\textbf{Community Insight:} ``{[}Include a final expert quote
  that encourages engagement, sharing, or community participation{]}''
  \emph{---{[}Expert Name{]}, {[}Title{]}}}

\section{References and Further
Reading}\label{references-and-further-reading}

\textbf{TEMPLATE NOTE:} A comprehensive references section with multiple
categories adds credibility and provides readers with clear learning
pathways. The dotfiles post demonstrated how 50+ well-organized sources
can elevate a technical blog post to professional standards.

\subsection{Academic Literature}\label{academic-literature}

\begin{enumerate}
\def\labelenumi{\arabic{enumi}.}
\tightlist
\item
  \textbf{Primary Research Papers:}

  \begin{itemize}
  \tightlist
  \item
    Wickham, H. (2014). ``Tidy Data''. \emph{Journal of Statistical
    Software}, 59(10), 1-23. https://doi.org/10.18637/jss.v059.i10
  \item
    Breiman, L. (2001). ``Random Forests''. \emph{Machine Learning},
    45(1), 5-32. https://doi.org/10.1023/A:1010933404324
  \item
    {[}Your domain-specific paper{]}. Author, A. (Year). ``Relevant
    Paper Title''. \emph{Journal Name}, Volume(Issue), pages. DOI
  \end{itemize}
\item
  \textbf{Foundational Books:}

  \begin{itemize}
  \tightlist
  \item
    Wickham, H., \& Grolemund, G. (2017). \emph{R for Data Science}.
    O'Reilly Media. https://r4ds.had.co.nz/
  \item
    James, G., Witten, D., Hastie, T., \& Tibshirani, R. (2021).
    \emph{An Introduction to Statistical Learning with Applications in
    R} (2nd ed.). Springer.
  \item
    {[}Your domain book{]}. Author, B. (Year). \emph{Book Title}.
    Publisher.
  \end{itemize}
\item
  \textbf{Statistical Methods:}

  \begin{itemize}
  \tightlist
  \item
    Box, G. E. P., Jenkins, G. M., Reinsel, G. C., \& Ljung, G. M.
    (2015). \emph{Time Series Analysis: Forecasting and Control} (5th
    ed.). Wiley.
  \item
    Gelman, A., \& Hill, J. (2006). \emph{Data Analysis Using Regression
    and Multilevel/Hierarchical Models}. Cambridge University Press.
  \end{itemize}
\end{enumerate}

\subsection{Blog Posts and Tutorials}\label{blog-posts-and-tutorials}

\begin{enumerate}
\def\labelenumi{\arabic{enumi}.}
\tightlist
\item
  \textbf{Technical Blog Posts:}

  \begin{itemize}
  \tightlist
  \item
    \href{https://www.r-bloggers.com/2023/01/advanced-ggplot2-techniques/}{R-bloggers:
    ``Advanced ggplot2 Techniques''} - Comprehensive visualization
    strategies
  \item
    \href{https://simplystatistics.org/posts/2023-02-15-statistics-data-science/}{Simply
    Statistics: ``The Role of Statistics in Data Science''} -
    Foundational concepts
  \item
    \href{https://towardsdatascience.com/machine-learning-best-practices-2023}{Towards
    Data Science: ``Machine Learning Best Practices''} - Practical
    implementation guidance
  \end{itemize}
\item
  \textbf{Package-Specific Tutorials:}

  \begin{itemize}
  \tightlist
  \item
    \href{https://example.com/package-intro}{Package creator's blog:
    ``Introduction to {[}PackageName{]}''} - Official guidance from
    package authors
  \item
    \href{https://www.rstudio.com/blog/}{RStudio Blog: ``New Features in
    {[}Package{]}''} - Updates and best practices
  \item
    \href{https://stackoverflow.com/questions/tagged/package-name}{Stack
    Overflow: ``Common {[}Package{]} Issues and Solutions''} - Community
    troubleshooting
  \end{itemize}
\item
  \textbf{Domain-Specific Applications:}

  \begin{itemize}
  \tightlist
  \item
    \href{https://example.com/industry-application}{Industry blog:
    ``Real-world Application of {[}Method{]}''} - Practical case studies
  \item
    \href{https://example.com/academic-perspective}{Academic blog:
    ``Methodological Considerations for {[}Technique{]}''} - Research
    perspectives
  \item
    \href{https://example.com/lessons-learned}{Practitioner blog:
    ``Lessons Learned from {[}Project{]}''} - Implementation insights
  \end{itemize}
\end{enumerate}

\subsection{Technical Documentation}\label{technical-documentation}

\begin{enumerate}
\def\labelenumi{\arabic{enumi}.}
\tightlist
\item
  \textbf{Package Documentation:}

  \begin{itemize}
  \tightlist
  \item
    \href{https://cran.r-project.org/package=packagename}{Package
    Reference Manual} - Complete function documentation
  \item
    \href{https://cran.r-project.org/package=packagename/vignettes/}{Package
    Vignettes} - Detailed usage examples
  \item
    \href{https://github.com/author/packagename}{GitHub Repository} -
    Source code and development issues
  \end{itemize}
\item
  \textbf{Language and Framework Guides:}

  \begin{itemize}
  \tightlist
  \item
    \href{https://cran.r-project.org/doc/manuals/r-release/R-lang.html}{R
    Language Definition} - Official R documentation
  \item
    \href{https://quarto.org/docs/}{Quarto Documentation} - Publishing
    framework reference
  \item
    \href{https://bookdown.org/yihui/rmarkdown-cookbook/}{RMarkdown
    Cookbook} - Advanced document preparation
  \end{itemize}
\item
  \textbf{Standards and Best Practices:}

  \begin{itemize}
  \tightlist
  \item
    \href{https://google.github.io/styleguide/Rguide.html}{Google's R
    Style Guide} - Code formatting standards
  \item
    \href{https://ropensci.org/packages/}{rOpenSci Packages} -
    Peer-reviewed R packages for research
  \item
    \href{https://cran.r-project.org/web/views/}{CRAN Task Views} -
    Domain-specific package collections
  \end{itemize}
\end{enumerate}

\subsection{Community Resources}\label{community-resources}

\begin{enumerate}
\def\labelenumi{\arabic{enumi}.}
\tightlist
\item
  \textbf{Q\&A and Discussion:}

  \begin{itemize}
  \tightlist
  \item
    \href{https://stats.stackexchange.com/}{Cross Validated} -
    Statistical methodology discussions
  \item
    \href{https://stackoverflow.com/questions/tagged/r}{Stack Overflow R
    Tag} - Programming troubleshooting
  \item
    \href{https://community.rstudio.com/}{RStudio Community} - User
    support and discussions
  \end{itemize}
\item
  \textbf{Social Learning:}

  \begin{itemize}
  \tightlist
  \item
    \href{https://twitter.com/hashtag/rstats}{\#rstats Twitter} -
    Community updates and tips
  \item
    \href{https://rweekly.org/}{R Weekly Newsletter} - Curated R news
    and resources
  \item
    \href{https://rladies.org/}{R-Ladies Global} - Inclusive R community
    and events
  \end{itemize}
\item
  \textbf{Professional Networks:}

  \begin{itemize}
  \tightlist
  \item
    \href{https://www.linkedin.com/groups/}{LinkedIn R Groups} -
    Professional networking and job opportunities
  \item
    \href{https://www.meetup.com/topics/r-project-for-statistical-computing/}{Meetup
    R Groups} - Local community events
  \item
    \href{https://user2024.r-project.org/}{UseR! Conference} - Annual R
    user conference
  \end{itemize}
\end{enumerate}

\subsection{Data Sources and
Repositories}\label{data-sources-and-repositories}

\begin{enumerate}
\def\labelenumi{\arabic{enumi}.}
\tightlist
\item
  \textbf{Public Datasets:}

  \begin{itemize}
  \tightlist
  \item
    \href{https://archive.ics.uci.edu/ml/}{UCI Machine Learning
    Repository} - Benchmark datasets
  \item
    \href{https://www.kaggle.com/datasets}{Kaggle Datasets} -
    Community-contributed data
  \item
    {[}government data portal{]} - Domain-specific public data
  \end{itemize}
\item
  \textbf{R Built-in Data:}

  \begin{itemize}
  \tightlist
  \item
    \texttt{datasets} package - Standard R datasets for examples
  \item
    {[}Your specific dataset source{]} - Domain-relevant data
    repositories
  \end{itemize}
\end{enumerate}

\subsection{Related Work and
Extensions}\label{related-work-and-extensions}

\begin{enumerate}
\def\labelenumi{\arabic{enumi}.}
\tightlist
\item
  \textbf{Methodological Extensions:}

  \begin{itemize}
  \tightlist
  \item
    Author, C. (Year). ``Extension of {[}Your Method{]}''.
    \emph{Journal}, Volume(Issue), pages.
  \item
    Author, D. (Year). ``Comparative Analysis of {[}Related
    Methods{]}''. \emph{Conference Proceedings}.
  \end{itemize}
\item
  \textbf{Applications in Other Domains:}

  \begin{itemize}
  \tightlist
  \item
    Author, E. (Year). ``Application to {[}Different Field{]}''.
    \emph{Domain Journal}, Volume(Issue), pages.
  \item
    Author, F. (Year). ``Cross-disciplinary Perspectives on
    {[}Topic{]}''. \emph{Interdisciplinary Journal}.
  \end{itemize}
\end{enumerate}

\begin{center}\rule{0.5\linewidth}{0.5pt}\end{center}

\textbf{Citation Note:} When using ideas or code from these resources,
please cite appropriately. For academic work, use standard citation
formats. For blog posts and online resources, include the author, title,
publication date, and URL. Consider using a bibliography file
(references.bib) for automated citation management with Quarto.

\section{Reproducibility Information}\label{reproducibility-information}

\subsection{Data Availability}\label{data-availability}

\begin{itemize}
\tightlist
\item
  \textbf{Dataset}: {[}Name and source of dataset used{]}
\item
  \textbf{Access}: {[}How others can access the data - URL, package,
  etc.{]}
\item
  \textbf{License}: {[}Data usage license and restrictions{]}
\end{itemize}

\subsection{Code Repository}\label{code-repository}

\begin{itemize}
\tightlist
\item
  \textbf{GitHub}: {[}Link to repository with complete analysis code{]}
\item
  \textbf{Commit}: {[}Specific commit hash for reproducibility{]}
\item
  \textbf{Environment}: {[}Docker image, renv lockfile, or environment
  specs{]}
\end{itemize}

\subsection{Session Information}\label{session-information}

\begin{verbatim}
R version 4.5.2 (2025-10-31)
Platform: aarch64-apple-darwin20
Running under: macOS Sequoia 15.6.1

Matrix products: default
BLAS:   /System/Library/Frameworks/Accelerate.framework/Versions/A/Frameworks/vecLib.framework/Versions/A/libBLAS.dylib 
LAPACK: /Library/Frameworks/R.framework/Versions/4.5-arm64/Resources/lib/libRlapack.dylib;  LAPACK version 3.12.1

locale:
[1] en_US.UTF-8/en_US.UTF-8/en_US.UTF-8/C/en_US.UTF-8/en_US.UTF-8

time zone: America/Los_Angeles
tzcode source: internal

attached base packages:
[1] stats     graphics  grDevices utils     datasets  methods   base     

loaded via a namespace (and not attached):
 [1] compiler_4.5.2  fastmap_1.2.0   cli_3.6.5       tools_4.5.2    
 [5] htmltools_0.5.9 parallel_4.5.2  yaml_2.3.11     rmarkdown_2.30 
 [9] knitr_1.50      jsonlite_2.0.0  xfun_0.54       digest_0.6.39  
[13] rlang_1.1.6     evaluate_1.0.5 
\end{verbatim}

\section{Appendix: {[}Optional Detailed
Information{]}}\label{appendix-optional-detailed-information}

\subsection{Appendix A: Complete Code}\label{appendix-a-complete-code}

\begin{Shaded}
\begin{Highlighting}[]
\CommentTok{\# Complete code for easy reproduction {-} replace with your actual code}
\CommentTok{\# library(your\_packages)}
\CommentTok{\# data \textless{}{-} load\_your\_data()}
\CommentTok{\# results \textless{}{-} your\_analysis(data)}
\CommentTok{\# plot(results)}
\end{Highlighting}
\end{Shaded}

\subsection{Appendix B: Mathematical
Details}\label{appendix-b-mathematical-details}

For statistical posts, include relevant formulas using LaTeX notation:

\textbf{Linear Regression Model:}
\[y_i = \beta_0 + \beta_1 x_{i1} + \beta_2 x_{i2} + ... + \beta_p x_{ip} + \epsilon_i\]

\textbf{Model Evaluation Metrics:} - \textbf{RMSE}:
\(RMSE = \sqrt{\frac{1}{n}\sum_{i=1}^{n}(y_i - \hat{y_i})^2}\) -
\textbf{R-squared}: \(R^2 = 1 - \frac{SS_{res}}{SS_{tot}}\)

{[}Additional mathematical explanations or derivations as needed{]}

\subsection{Appendix C: Additional
Data}\label{appendix-c-additional-data}

{[}Additional tables, charts, or data summaries{]}

\begin{center}\rule{0.5\linewidth}{0.5pt}\end{center}

\subsection{Share This Post}\label{share-this-post}

Found this helpful? Share it with your network:

\begin{itemize}
\tightlist
\item
  \href{https://twitter.com/intent/tweet?text=Check\%20out\%20this\%20analysis&url=YOUR_POST_URL&via=rgt47}{Twitter}
\item
  \href{https://www.linkedin.com/sharing/share-offsite/?url=YOUR_POST_URL}{LinkedIn}
\item
  \href{https://reddit.com/submit?url=YOUR_POST_URL&title=YOUR_POST_TITLE}{Reddit}
\end{itemize}

\subsection{Connect and Discuss}\label{connect-and-discuss}

\emph{Have questions or suggestions? I'd love to hear from you:}

\begin{itemize}
\tightlist
\item
  \textbf{Twitter}: \href{https://twitter.com/rgt47}{\citet{rgt47}} -
  Quick questions and discussions
\item
  \textbf{LinkedIn}: \href{https://linkedin.com/in/rgthomaslab}{Ronald
  Glenn Thomas} - Professional networking
\item
  \textbf{GitHub}: \href{https://github.com/rgt47}{rgt47} - Code,
  issues, and contributions
\item
  \textbf{Email}: \href{https://rgtlab.org/contact}{Contact through
  website} - Detailed inquiries
\end{itemize}

\emph{Comments are enabled below via Utterances - join the discussion!}

\begin{center}\rule{0.5\linewidth}{0.5pt}\end{center}

\subsection{About the Author}\label{about-the-author}

\textbf{Ronald (Ryy) Glenn Thomas} is a biostatistician and data
scientist at UC San Diego, specializing in statistical computing,
machine learning applications in healthcare, and reproducible research
methods. He develops R packages and conducts research at the
intersection of statistics, data science, and clinical research.

\emph{Connect: \href{https://rgtlab.org}{Website} \textbar{}
\href{https://orcid.org/0000-0003-1686-4965}{ORCID} \textbar{}
\href{https://scholar.google.com/citations?user=YOUR_ID}{Google
Scholar}}


\bibliography{references.bib}



\end{document}
