% Options for packages loaded elsewhere
% Options for packages loaded elsewhere
\PassOptionsToPackage{unicode}{hyperref}
\PassOptionsToPackage{hyphens}{url}
\PassOptionsToPackage{dvipsnames,svgnames,x11names}{xcolor}
%
\documentclass[
  letterpaper,
  DIV=11,
  numbers=noendperiod]{scrartcl}
\usepackage{xcolor}
\usepackage{amsmath,amssymb}
\setcounter{secnumdepth}{5}
\usepackage{iftex}
\ifPDFTeX
  \usepackage[T1]{fontenc}
  \usepackage[utf8]{inputenc}
  \usepackage{textcomp} % provide euro and other symbols
\else % if luatex or xetex
  \usepackage{unicode-math} % this also loads fontspec
  \defaultfontfeatures{Scale=MatchLowercase}
  \defaultfontfeatures[\rmfamily]{Ligatures=TeX,Scale=1}
\fi
\usepackage{lmodern}
\ifPDFTeX\else
  % xetex/luatex font selection
\fi
% Use upquote if available, for straight quotes in verbatim environments
\IfFileExists{upquote.sty}{\usepackage{upquote}}{}
\IfFileExists{microtype.sty}{% use microtype if available
  \usepackage[]{microtype}
  \UseMicrotypeSet[protrusion]{basicmath} % disable protrusion for tt fonts
}{}
\makeatletter
\@ifundefined{KOMAClassName}{% if non-KOMA class
  \IfFileExists{parskip.sty}{%
    \usepackage{parskip}
  }{% else
    \setlength{\parindent}{0pt}
    \setlength{\parskip}{6pt plus 2pt minus 1pt}}
}{% if KOMA class
  \KOMAoptions{parskip=half}}
\makeatother
% Make \paragraph and \subparagraph free-standing
\makeatletter
\ifx\paragraph\undefined\else
  \let\oldparagraph\paragraph
  \renewcommand{\paragraph}{
    \@ifstar
      \xxxParagraphStar
      \xxxParagraphNoStar
  }
  \newcommand{\xxxParagraphStar}[1]{\oldparagraph*{#1}\mbox{}}
  \newcommand{\xxxParagraphNoStar}[1]{\oldparagraph{#1}\mbox{}}
\fi
\ifx\subparagraph\undefined\else
  \let\oldsubparagraph\subparagraph
  \renewcommand{\subparagraph}{
    \@ifstar
      \xxxSubParagraphStar
      \xxxSubParagraphNoStar
  }
  \newcommand{\xxxSubParagraphStar}[1]{\oldsubparagraph*{#1}\mbox{}}
  \newcommand{\xxxSubParagraphNoStar}[1]{\oldsubparagraph{#1}\mbox{}}
\fi
\makeatother

\usepackage{color}
\usepackage{fancyvrb}
\newcommand{\VerbBar}{|}
\newcommand{\VERB}{\Verb[commandchars=\\\{\}]}
\DefineVerbatimEnvironment{Highlighting}{Verbatim}{commandchars=\\\{\}}
% Add ',fontsize=\small' for more characters per line
\usepackage{framed}
\definecolor{shadecolor}{RGB}{241,243,245}
\newenvironment{Shaded}{\begin{snugshade}}{\end{snugshade}}
\newcommand{\AlertTok}[1]{\textcolor[rgb]{0.68,0.00,0.00}{#1}}
\newcommand{\AnnotationTok}[1]{\textcolor[rgb]{0.37,0.37,0.37}{#1}}
\newcommand{\AttributeTok}[1]{\textcolor[rgb]{0.40,0.45,0.13}{#1}}
\newcommand{\BaseNTok}[1]{\textcolor[rgb]{0.68,0.00,0.00}{#1}}
\newcommand{\BuiltInTok}[1]{\textcolor[rgb]{0.00,0.23,0.31}{#1}}
\newcommand{\CharTok}[1]{\textcolor[rgb]{0.13,0.47,0.30}{#1}}
\newcommand{\CommentTok}[1]{\textcolor[rgb]{0.37,0.37,0.37}{#1}}
\newcommand{\CommentVarTok}[1]{\textcolor[rgb]{0.37,0.37,0.37}{\textit{#1}}}
\newcommand{\ConstantTok}[1]{\textcolor[rgb]{0.56,0.35,0.01}{#1}}
\newcommand{\ControlFlowTok}[1]{\textcolor[rgb]{0.00,0.23,0.31}{\textbf{#1}}}
\newcommand{\DataTypeTok}[1]{\textcolor[rgb]{0.68,0.00,0.00}{#1}}
\newcommand{\DecValTok}[1]{\textcolor[rgb]{0.68,0.00,0.00}{#1}}
\newcommand{\DocumentationTok}[1]{\textcolor[rgb]{0.37,0.37,0.37}{\textit{#1}}}
\newcommand{\ErrorTok}[1]{\textcolor[rgb]{0.68,0.00,0.00}{#1}}
\newcommand{\ExtensionTok}[1]{\textcolor[rgb]{0.00,0.23,0.31}{#1}}
\newcommand{\FloatTok}[1]{\textcolor[rgb]{0.68,0.00,0.00}{#1}}
\newcommand{\FunctionTok}[1]{\textcolor[rgb]{0.28,0.35,0.67}{#1}}
\newcommand{\ImportTok}[1]{\textcolor[rgb]{0.00,0.46,0.62}{#1}}
\newcommand{\InformationTok}[1]{\textcolor[rgb]{0.37,0.37,0.37}{#1}}
\newcommand{\KeywordTok}[1]{\textcolor[rgb]{0.00,0.23,0.31}{\textbf{#1}}}
\newcommand{\NormalTok}[1]{\textcolor[rgb]{0.00,0.23,0.31}{#1}}
\newcommand{\OperatorTok}[1]{\textcolor[rgb]{0.37,0.37,0.37}{#1}}
\newcommand{\OtherTok}[1]{\textcolor[rgb]{0.00,0.23,0.31}{#1}}
\newcommand{\PreprocessorTok}[1]{\textcolor[rgb]{0.68,0.00,0.00}{#1}}
\newcommand{\RegionMarkerTok}[1]{\textcolor[rgb]{0.00,0.23,0.31}{#1}}
\newcommand{\SpecialCharTok}[1]{\textcolor[rgb]{0.37,0.37,0.37}{#1}}
\newcommand{\SpecialStringTok}[1]{\textcolor[rgb]{0.13,0.47,0.30}{#1}}
\newcommand{\StringTok}[1]{\textcolor[rgb]{0.13,0.47,0.30}{#1}}
\newcommand{\VariableTok}[1]{\textcolor[rgb]{0.07,0.07,0.07}{#1}}
\newcommand{\VerbatimStringTok}[1]{\textcolor[rgb]{0.13,0.47,0.30}{#1}}
\newcommand{\WarningTok}[1]{\textcolor[rgb]{0.37,0.37,0.37}{\textit{#1}}}

\usepackage{longtable,booktabs,array}
\usepackage{calc} % for calculating minipage widths
% Correct order of tables after \paragraph or \subparagraph
\usepackage{etoolbox}
\makeatletter
\patchcmd\longtable{\par}{\if@noskipsec\mbox{}\fi\par}{}{}
\makeatother
% Allow footnotes in longtable head/foot
\IfFileExists{footnotehyper.sty}{\usepackage{footnotehyper}}{\usepackage{footnote}}
\makesavenoteenv{longtable}
\usepackage{graphicx}
\makeatletter
\newsavebox\pandoc@box
\newcommand*\pandocbounded[1]{% scales image to fit in text height/width
  \sbox\pandoc@box{#1}%
  \Gscale@div\@tempa{\textheight}{\dimexpr\ht\pandoc@box+\dp\pandoc@box\relax}%
  \Gscale@div\@tempb{\linewidth}{\wd\pandoc@box}%
  \ifdim\@tempb\p@<\@tempa\p@\let\@tempa\@tempb\fi% select the smaller of both
  \ifdim\@tempa\p@<\p@\scalebox{\@tempa}{\usebox\pandoc@box}%
  \else\usebox{\pandoc@box}%
  \fi%
}
% Set default figure placement to htbp
\def\fps@figure{htbp}
\makeatother





\setlength{\emergencystretch}{3em} % prevent overfull lines

\providecommand{\tightlist}{%
  \setlength{\itemsep}{0pt}\setlength{\parskip}{0pt}}



 


\KOMAoption{captions}{tableheading}
\makeatletter
\@ifpackageloaded{tcolorbox}{}{\usepackage[skins,breakable]{tcolorbox}}
\@ifpackageloaded{fontawesome5}{}{\usepackage{fontawesome5}}
\definecolor{quarto-callout-color}{HTML}{909090}
\definecolor{quarto-callout-note-color}{HTML}{0758E5}
\definecolor{quarto-callout-important-color}{HTML}{CC1914}
\definecolor{quarto-callout-warning-color}{HTML}{EB9113}
\definecolor{quarto-callout-tip-color}{HTML}{00A047}
\definecolor{quarto-callout-caution-color}{HTML}{FC5300}
\definecolor{quarto-callout-color-frame}{HTML}{acacac}
\definecolor{quarto-callout-note-color-frame}{HTML}{4582ec}
\definecolor{quarto-callout-important-color-frame}{HTML}{d9534f}
\definecolor{quarto-callout-warning-color-frame}{HTML}{f0ad4e}
\definecolor{quarto-callout-tip-color-frame}{HTML}{02b875}
\definecolor{quarto-callout-caution-color-frame}{HTML}{fd7e14}
\makeatother
\makeatletter
\@ifpackageloaded{caption}{}{\usepackage{caption}}
\AtBeginDocument{%
\ifdefined\contentsname
  \renewcommand*\contentsname{Table of contents}
\else
  \newcommand\contentsname{Table of contents}
\fi
\ifdefined\listfigurename
  \renewcommand*\listfigurename{List of Figures}
\else
  \newcommand\listfigurename{List of Figures}
\fi
\ifdefined\listtablename
  \renewcommand*\listtablename{List of Tables}
\else
  \newcommand\listtablename{List of Tables}
\fi
\ifdefined\figurename
  \renewcommand*\figurename{Figure}
\else
  \newcommand\figurename{Figure}
\fi
\ifdefined\tablename
  \renewcommand*\tablename{Table}
\else
  \newcommand\tablename{Table}
\fi
}
\@ifpackageloaded{float}{}{\usepackage{float}}
\floatstyle{ruled}
\@ifundefined{c@chapter}{\newfloat{codelisting}{h}{lop}}{\newfloat{codelisting}{h}{lop}[chapter]}
\floatname{codelisting}{Listing}
\newcommand*\listoflistings{\listof{codelisting}{List of Listings}}
\makeatother
\makeatletter
\makeatother
\makeatletter
\@ifpackageloaded{caption}{}{\usepackage{caption}}
\@ifpackageloaded{subcaption}{}{\usepackage{subcaption}}
\makeatother
\usepackage{bookmark}
\IfFileExists{xurl.sty}{\usepackage{xurl}}{} % add URL line breaks if available
\urlstyle{same}
\hypersetup{
  pdftitle={From Markdown to Blog Post: Converting Documentation into a Reproducible ZZCOLLAB Workflow},
  pdfauthor={Ronald G. Thomas},
  colorlinks=true,
  linkcolor={blue},
  filecolor={Maroon},
  citecolor={Blue},
  urlcolor={Blue},
  pdfcreator={LaTeX via pandoc}}


\title{From Markdown to Blog Post: Converting Documentation into a
Reproducible ZZCOLLAB Workflow}
\usepackage{etoolbox}
\makeatletter
\providecommand{\subtitle}[1]{% add subtitle to \maketitle
  \apptocmd{\@title}{\par {\large #1 \par}}{}{}
}
\makeatother
\subtitle{A complete guide to integrating technical reference materials
into your Quarto blog system}
\author{Ronald G. Thomas}
\date{2025-12-02}
\begin{document}
\maketitle

\renewcommand*\contentsname{Table of contents}
{
\hypersetup{linkcolor=}
\setcounter{tocdepth}{3}
\tableofcontents
}

\begin{figure}[H]

{\centering \pandocbounded{\includegraphics[keepaspectratio]{media/images/markdown-to-blog-concept.jpg}}

}

\caption{Converting your documentation into a professional blog post}

\end{figure}%

\emph{A structured workflow for publishing technical documentation with
reproducibility built-in}

\section{Introduction}\label{introduction}

You have excellent technical documentation in markdown format. It
explains complex processes clearly, includes working code examples, and
provides valuable guidance. But it sits as a standalone file on your
disk, never reaching the audience that could benefit from it.

This post documents how to transform that markdown documentation into a
professional, publishable blog post integrated with your Quarto blog
system---while maintaining reproducibility through ZZCOLLAB and Docker.

\subsection{What You'll Learn}\label{what-youll-learn}

\begin{itemize}
\tightlist
\item
  \textbf{The complete workflow}: Step-by-step process for converting
  markdown to blog post
\item
  \textbf{Key decisions}: How to structure content for blog format
\item
  \textbf{Technical integration}: Symlinks, Quarto compatibility, and
  git workflow
\item
  \textbf{Verification steps}: Testing and validation before publication
\item
  \textbf{Troubleshooting}: Common issues and solutions
\item
  \textbf{Timeline}: Realistic expectations (40 minutes total)
\end{itemize}

\subsection{Why This Matters}\label{why-this-matters}

Blog posts discoverable through search and social sharing reach a
broader audience than markdown files in your project directories. Using
ZZCOLLAB ensures:

\begin{itemize}
\tightlist
\item
  \textbf{Reproducibility} - Docker + renv lock exact environment
\item
  \textbf{Discoverability} - Integrated with your blog's search and
  tagging
\item
  \textbf{Professionalism} - Proper metadata, structure, and styling
\item
  \textbf{Maintainability} - Version controlled and easily updated
\item
  \textbf{Extensibility} - Can add data analysis, visualizations, or
  interactive elements later
\end{itemize}

\section{Part 1: Prerequisites and
Planning}\label{part-1-prerequisites-and-planning}

\subsection{Tools You'll Need}\label{tools-youll-need}

\textbf{Must Have}: - ZZCOLLAB framework (installed) - Quarto (for
rendering) - Git (version control) - Bash shell

\textbf{Nice to Have}: - GitHub CLI (\texttt{gh}) for automation -
Docker (for full reproducibility) - Local Quarto preview capability

\subsection{Initial Assessment: 5
Minutes}\label{initial-assessment-5-minutes}

Before starting, evaluate your source markdown:

\textbf{Ask Yourself}: 1. How long is it? (200 lines? 600 lines?) 2.
Does it have code examples? (bash, R, Python?) 3. Does it need images?
(hero image, embedded diagrams?) 4. Is it a tutorial, reference guide,
or how-to? 5. What's the target audience?

\textbf{Example: GitHub Archive Post}

\begin{verbatim}
- Length: ~600 lines (substantial)
- Code: Yes (bash script)
- Images: Yes (hero + concept)
- Type: How-to guide + reference
- Audience: DevOps engineers
- Estimate: 10 minutes to convert content
\end{verbatim}

\subsection{Make Key Decisions}\label{make-key-decisions}

Before initializing the project, decide:

\begin{longtable}[]{@{}
  >{\raggedright\arraybackslash}p{(\linewidth - 4\tabcolsep) * \real{0.2564}}
  >{\raggedright\arraybackslash}p{(\linewidth - 4\tabcolsep) * \real{0.4615}}
  >{\raggedright\arraybackslash}p{(\linewidth - 4\tabcolsep) * \real{0.2821}}@{}}
\toprule\noalign{}
\begin{minipage}[b]{\linewidth}\raggedright
Decision
\end{minipage} & \begin{minipage}[b]{\linewidth}\raggedright
GitHub Archive
\end{minipage} & \begin{minipage}[b]{\linewidth}\raggedright
Your Post
\end{minipage} \\
\midrule\noalign{}
\endhead
\bottomrule\noalign{}
\endlastfoot
\textbf{Blog slug} & \texttt{github\_archive} &
\texttt{{[}kebab-case-name{]}} \\
\textbf{ZZCOLLAB profile} & \texttt{ubuntu\_standard\_publishing} & Same
(has Quarto) \\
\textbf{Include analysis?} & No & Yes / No \\
\textbf{Media assets} & 2 images & How many? \\
\textbf{Categories} & DevOps, Git, etc & ? \\
\end{longtable}

\section{Part 2: Create the ZZCOLLAB Blog
Project}\label{part-2-create-the-zzcollab-blog-project}

\subsection{Step 1: Create Directory}\label{step-1-create-directory}

\begin{Shaded}
\begin{Highlighting}[]
\BuiltInTok{cd}\NormalTok{ \textasciitilde{}/prj/qblog/posts}
\FunctionTok{mkdir}\NormalTok{ my\_blog\_post }\KeywordTok{\&\&} \BuiltInTok{cd}\NormalTok{ my\_blog\_post}
\end{Highlighting}
\end{Shaded}

Replace \texttt{my\_blog\_post} with your kebab-case slug (e.g.,
\texttt{github\_archive}, \texttt{python\_async},
\texttt{data\_pipeline}).

\subsection{Step 2: Initialize
ZZCOLLAB}\label{step-2-initialize-zzcollab}

\begin{Shaded}
\begin{Highlighting}[]
\ExtensionTok{zzcollab} \AttributeTok{{-}r}\NormalTok{ ubuntu\_standard\_publishing}
\end{Highlighting}
\end{Shaded}

This creates: - Standard project structure (analysis/, R/, tests/) -
Dockerfile (reproducible environment) - renv.lock (R package versions) -
Makefile (build automation) - .zzcollab/manifest.json (project metadata)

\textbf{Expected output}: \textasciitilde20 lines of checkmarks showing
successful initialization

\subsection{Step 3: Set Up Blog
Structure}\label{step-3-set-up-blog-structure}

\begin{Shaded}
\begin{Highlighting}[]
\ExtensionTok{./modules/setup\_symlinks.sh}
\end{Highlighting}
\end{Shaded}

This creates the \textbf{dual-symlink system} that makes everything
work:

\textbf{At post root} (for Quarto):

\begin{verbatim}
index.qmd → analysis/paper/index.qmd
figures/  → analysis/figures/
media/    → analysis/media/
data/     → analysis/data/
\end{verbatim}

\textbf{In analysis/paper/} (for editing):

\begin{verbatim}
figures/ → ../figures/
media/   → ../media/
data/    → ../data/
\end{verbatim}

\textbf{Why symlinks?} - Quarto expects \texttt{posts/*/index.qmd} at
root - ZZCOLLAB/rrtools puts content in \texttt{analysis/paper/} -
Symlinks bridge both conventions - Image paths like
\texttt{!{[}{]}(media/images/hero.jpg)} work automatically

\subsection{Step 4: Verify Structure}\label{step-4-verify-structure}

\begin{Shaded}
\begin{Highlighting}[]
\CommentTok{\# Check root symlinks}
\FunctionTok{ls} \AttributeTok{{-}la} \KeywordTok{|} \FunctionTok{grep} \StringTok{"\^{}l"}

\CommentTok{\# Should show all 4 symlinks ✓}
\CommentTok{\# index.qmd → analysis/paper/index.qmd}
\CommentTok{\# figures → analysis/figures}
\CommentTok{\# media → analysis/media}
\CommentTok{\# data → analysis/data}

\CommentTok{\# Check paper directory}
\FunctionTok{ls} \AttributeTok{{-}la}\NormalTok{ analysis/paper/}

\CommentTok{\# Should show:}
\CommentTok{\# figures → ../figures}
\CommentTok{\# media → ../media}
\CommentTok{\# data → ../data}
\CommentTok{\# index.qmd (actual file)}
\end{Highlighting}
\end{Shaded}

\section{Part 3: Convert Markdown to Quarto Blog
Post}\label{part-3-convert-markdown-to-quarto-blog-post}

\subsection{Create Proper YAML
Frontmatter}\label{create-proper-yaml-frontmatter}

Replace the template YAML with your blog post metadata:

\begin{Shaded}
\begin{Highlighting}[]
\PreprocessorTok{{-}{-}{-}}
\FunctionTok{title}\KeywordTok{:}\AttributeTok{ }\StringTok{"Your Blog Post Title"}
\FunctionTok{subtitle}\KeywordTok{:}\AttributeTok{ }\StringTok{"Optional subtitle (2{-}3 words)"}
\FunctionTok{author}\KeywordTok{:}\AttributeTok{ }\StringTok{"Your Name"}
\FunctionTok{date}\KeywordTok{:}\AttributeTok{ }\StringTok{"2025{-}12{-}02"}
\FunctionTok{categories}\KeywordTok{:}\AttributeTok{ }\KeywordTok{[}\AttributeTok{Category1}\KeywordTok{,}\AttributeTok{ Category2}\KeywordTok{,}\AttributeTok{ Category3}\KeywordTok{]}
\FunctionTok{description}\KeywordTok{:}\AttributeTok{ }\StringTok{"2{-}3 sentence summary of the post"}
\FunctionTok{image}\KeywordTok{:}\AttributeTok{ }\StringTok{"media/images/hero{-}image.jpg"}
\FunctionTok{document{-}type}\KeywordTok{:}\AttributeTok{ }\StringTok{"blog"}
\FunctionTok{draft}\KeywordTok{:}\AttributeTok{ }\CharTok{false}
\FunctionTok{execute}\KeywordTok{:}
\AttributeTok{  }\FunctionTok{echo}\KeywordTok{:}\AttributeTok{ }\CharTok{true}
\AttributeTok{  }\FunctionTok{warning}\KeywordTok{:}\AttributeTok{ }\CharTok{false}
\AttributeTok{  }\FunctionTok{message}\KeywordTok{:}\AttributeTok{ }\CharTok{false}
\FunctionTok{format}\KeywordTok{:}
\AttributeTok{  }\FunctionTok{html}\KeywordTok{:}
\AttributeTok{    }\FunctionTok{toc}\KeywordTok{:}\AttributeTok{ }\CharTok{true}
\AttributeTok{    }\FunctionTok{toc{-}depth}\KeywordTok{:}\AttributeTok{ }\DecValTok{3}
\AttributeTok{    }\FunctionTok{code{-}fold}\KeywordTok{:}\AttributeTok{ }\CharTok{false}
\AttributeTok{    }\FunctionTok{code{-}tools}\KeywordTok{:}\AttributeTok{ }\CharTok{false}
\PreprocessorTok{{-}{-}{-}}
\end{Highlighting}
\end{Shaded}

\textbf{Key fields}: - \texttt{title}: 50-70 characters, searchable,
descriptive - \texttt{categories}: 2-4 relevant categories for filtering
- \texttt{description}: Appears in blog listing, capture value
proposition - \texttt{draft:\ false}: Set to true to hide from
publication - \texttt{date}: Publish date (or \texttt{last-modified} for
auto-update) - \texttt{code-fold:\ false}: Show code by default (good
for tutorials)

\subsection{Structure Your Content}\label{structure-your-content}

\textbf{Blog post structure} (proven effective):

\begin{verbatim}
1. Hero image + subtitle (1 paragraph)
2. Introduction (2-3 paragraphs)
   - What this post covers
   - Why it matters
   - What they'll learn
3. Problem statement or background (2-3 sections)
4. Solution/main content (multiple detailed sections)
5. Code examples or implementation (complete, runnable)
6. Usage instructions (step-by-step)
7. Examples with actual output
8. Best practices (tips and recommendations)
9. Troubleshooting (common issues)
10. Key takeaways (callout box summary)
11. Further reading (links to related docs)
\end{verbatim}

\subsection{Adapt Your Markdown}\label{adapt-your-markdown}

\textbf{Markdown → Quarto conversions}:

\begin{Shaded}
\begin{Highlighting}[]
\FunctionTok{\# Simple heading → stays the same}
\FunctionTok{\#\# Sub{-}heading → stays the same}

\CommentTok{[}\OtherTok{Link text}\CommentTok{](url)}\NormalTok{ → stays the same}
\AlertTok{![Alt text](path)}\NormalTok{ → stays the same}
\SpecialStringTok{{-} }\NormalTok{Bullet → stays the same}

\AttributeTok{\textgreater{} Blockquote → becomes :::callout{-}note}
\end{Highlighting}
\end{Shaded}

\textbf{Special: Convert blockquotes to Quarto callouts}

\begin{Shaded}
\begin{Highlighting}[]
\NormalTok{::: \{.callout{-}note\}}
\NormalTok{\#\# Summary}

\NormalTok{Your key takeaway goes here}
\NormalTok{:::}
\end{Highlighting}
\end{Shaded}

\subsection{Code Blocks Matter}\label{code-blocks-matter}

Make code blocks production-ready:

\begin{Shaded}
\begin{Highlighting}[]
\CommentTok{\#!/bin/bash}
\CommentTok{\# Complete, runnable example}
\CommentTok{\# Comments explaining non{-}obvious parts}
\BuiltInTok{set} \AttributeTok{{-}e}  \CommentTok{\# Exit on error}

\CommentTok{\# Actual working code}
\ExtensionTok{your\_command} \AttributeTok{{-}{-}with{-}options}
\end{Highlighting}
\end{Shaded}

\textbf{Best practices}: - Specify language (bash, r, python, sql, etc)
- Make examples self-contained - Include comments for clarity - Test
them before publishing

\section{Part 4: Add Media Assets}\label{part-4-add-media-assets}

\subsection{Prepare Images}\label{prepare-images}

\begin{Shaded}
\begin{Highlighting}[]
\CommentTok{\# Copy your hero image}
\FunctionTok{cp}\NormalTok{ \textasciitilde{}/path/to/hero{-}image.jpg analysis/media/images/my{-}post{-}hero.jpg}

\CommentTok{\# Copy other images (concept art, diagrams, etc)}
\FunctionTok{cp}\NormalTok{ \textasciitilde{}/path/to/concept{-}image.png analysis/media/images/my{-}post{-}concept.png}
\end{Highlighting}
\end{Shaded}

\subsection{Document Image Sources}\label{document-image-sources}

Create a README in the images directory:

\begin{Shaded}
\begin{Highlighting}[]
\FunctionTok{cat} \OperatorTok{\textgreater{}}\NormalTok{ analysis/media/images/README.md }\OperatorTok{\textless{}\textless{} \textquotesingle{}EOF\textquotesingle{}}
\StringTok{\# Image Sources and Attribution}

\StringTok{\#\# my{-}post{-}hero.jpg}
\StringTok{{-} Source: Unsplash / Pixabay / Pexels}
\StringTok{{-} Creator: [Creator Name if known]}
\StringTok{{-} License: [License type]}
\StringTok{{-} URL: [Link to original if available]}

\StringTok{\#\# my{-}post{-}concept.png}
\StringTok{{-} Source: [Where it came from]}
\StringTok{{-} License: [CC0, CC{-}BY, etc]}
\StringTok{{-} Notes: [Any other relevant info]}
\OperatorTok{EOF}
\end{Highlighting}
\end{Shaded}

\textbf{Why document sources?} - Respect creator attribution - Document
licensing compliance - Make it easy to replace if needed - Good practice
for professional blogs

\subsection{Reference Images in Blog
Post}\label{reference-images-in-blog-post}

\begin{Shaded}
\begin{Highlighting}[]
\NormalTok{![Descriptive alt text](media/images/my{-}post{-}hero.jpg)\{.img{-}fluid\}}

\NormalTok{*Subtitle or caption for the image*}
\end{Highlighting}
\end{Shaded}

The \texttt{\{.img-fluid\}} class makes images responsive on mobile.

\section{Part 5: Update Project
Metadata}\label{part-5-update-project-metadata}

\subsection{Customize README.md}\label{customize-readme.md}

The ZZCOLLAB template creates a generic README. Personalize it:

\begin{Shaded}
\begin{Highlighting}[]
\FunctionTok{\# Your Blog Post Title}

\AttributeTok{\textgreater{} A blog post about }\CommentTok{[}\OtherTok{topic}\CommentTok{]}

\FunctionTok{\#\# Quick Start}

\NormalTok{To read this blog post:}

\InformationTok{\textasciigrave{}\textasciigrave{}\textasciigrave{}bash}
\ExtensionTok{quarto}\NormalTok{ render analysis/paper/index.qmd}
\ExtensionTok{open}\NormalTok{ index.html}
\end{Highlighting}
\end{Shaded}

\subsection{Structure}\label{structure}

\begin{itemize}
\tightlist
\item
  \texttt{analysis/paper/index.qmd} - Main blog post content
\item
  \texttt{analysis/media/images/} - Hero image and diagrams
\item
  \texttt{Dockerfile} - Reproducible environment
\item
  \texttt{renv.lock} - R package versions
\end{itemize}

\subsection{To Extend This Post}\label{to-extend-this-post}

Add R analysis:

\begin{Shaded}
\begin{Highlighting}[]
\ExtensionTok{Rscript}\NormalTok{ analysis/scripts/01\_prepare\_data.R}
\ExtensionTok{Rscript}\NormalTok{ analysis/scripts/02\_generate\_figures.R}
\ExtensionTok{quarto}\NormalTok{ render analysis/paper/index.qmd}
\end{Highlighting}
\end{Shaded}

\begin{verbatim}

## Update DESCRIPTION File

```r
Package: myblogpost
Title: Your Blog Post Title Here
Description: Brief description of what the post covers
Version: 1.0.0
Authors@R: person("Your Name", role = c("aut", "cre"))
\end{verbatim}

\section{Part 6: Test and Validate}\label{part-6-test-and-validate}

\subsection{Quick Validation}\label{quick-validation}

\begin{Shaded}
\begin{Highlighting}[]
\CommentTok{\# Verify symlinks are correct}
\FunctionTok{ls} \AttributeTok{{-}la}\NormalTok{ index.qmd}
\CommentTok{\# Should show: index.qmd {-}\textgreater{} analysis/paper/index.qmd ✓}

\CommentTok{\# Check YAML syntax (proper indentation, no tabs)}
\FunctionTok{head} \AttributeTok{{-}25}\NormalTok{ analysis/paper/index.qmd}

\CommentTok{\# Verify image paths are relative}
\FunctionTok{grep} \StringTok{"images/"}\NormalTok{ analysis/paper/index.qmd}
\CommentTok{\# Should show: media/images/filename.jpg (not absolute path)}
\end{Highlighting}
\end{Shaded}

\subsection{Render Locally (If Quarto
Installed)}\label{render-locally-if-quarto-installed}

\begin{Shaded}
\begin{Highlighting}[]
\BuiltInTok{cd}\NormalTok{ \textasciitilde{}/prj/qblog/posts/my\_blog\_post}
\ExtensionTok{quarto}\NormalTok{ render analysis/paper/index.qmd}

\CommentTok{\# Preview in browser}
\ExtensionTok{open}\NormalTok{ index.html}
\end{Highlighting}
\end{Shaded}

\textbf{Common issues}:

\begin{longtable}[]{@{}
  >{\raggedright\arraybackslash}p{(\linewidth - 4\tabcolsep) * \real{0.3684}}
  >{\raggedright\arraybackslash}p{(\linewidth - 4\tabcolsep) * \real{0.3684}}
  >{\raggedright\arraybackslash}p{(\linewidth - 4\tabcolsep) * \real{0.2632}}@{}}
\toprule\noalign{}
\begin{minipage}[b]{\linewidth}\raggedright
Error
\end{minipage} & \begin{minipage}[b]{\linewidth}\raggedright
Cause
\end{minipage} & \begin{minipage}[b]{\linewidth}\raggedright
Fix
\end{minipage} \\
\midrule\noalign{}
\endhead
\bottomrule\noalign{}
\endlastfoot
\texttt{index.qmd\ not\ found} & Symlink broken & Check:
\texttt{ls\ -la\ index.qmd} \\
\texttt{Image\ not\ found} & Wrong path & Use relative:
\texttt{media/images/file.jpg} \\
\texttt{YAML\ parse\ error} & Indentation (tabs?) & Use spaces only \\
Code not highlighted & Wrong language & Specify:
\texttt{\textasciigrave{}\textasciigrave{}\textasciigrave{}bash} not
\texttt{\textasciigrave{}\textasciigrave{}\textasciigrave{}} \\
\end{longtable}

\section{Part 7: Integration with Parent
Blog}\label{part-7-integration-with-parent-blog}

Your parent Quarto blog's \texttt{\_quarto.yml} should already have:

\begin{Shaded}
\begin{Highlighting}[]
\FunctionTok{project}\KeywordTok{:}
\AttributeTok{  }\FunctionTok{type}\KeywordTok{:}\AttributeTok{ website}
\AttributeTok{  }\FunctionTok{render}\KeywordTok{:}
\AttributeTok{    }\KeywordTok{{-}}\AttributeTok{ }\StringTok{"posts/*/index.qmd"}
\end{Highlighting}
\end{Shaded}

\textbf{Great news}: Because we used symlinks, Quarto automatically
finds your post!

\textbf{No changes needed to parent blog configuration.}

When you render the parent blog:

\begin{Shaded}
\begin{Highlighting}[]
\BuiltInTok{cd}\NormalTok{ \textasciitilde{}/prj/qblog}
\ExtensionTok{quarto}\NormalTok{ render}
\end{Highlighting}
\end{Shaded}

It discovers your post via the \texttt{posts/my\_blog\_post/index.qmd}
symlink and automatically includes it in the site.

\section{Part 8: Version Control}\label{part-8-version-control}

\subsection{Commit Your Work}\label{commit-your-work}

\begin{Shaded}
\begin{Highlighting}[]
\BuiltInTok{cd}\NormalTok{ \textasciitilde{}/prj/qblog/posts/my\_blog\_post}

\CommentTok{\# Stage all changes}
\FunctionTok{git}\NormalTok{ add .}

\CommentTok{\# Check what you\textquotesingle{}re committing}
\FunctionTok{git}\NormalTok{ status}

\CommentTok{\# Commit with descriptive message}
\FunctionTok{git}\NormalTok{ commit }\AttributeTok{{-}m} \StringTok{"Add blog post: Your Post Title}

\StringTok{{-} analysis/paper/index.qmd: Main content}
\StringTok{{-} analysis/media/images/: Hero and supporting images}
\StringTok{{-} README.md: Post documentation}
\StringTok{{-} Docker/renv: Reproducible environment}

\StringTok{Topics covered:}
\StringTok{{-} [Main topic 1]}
\StringTok{{-} [Main topic 2]}
\StringTok{{-} [Main topic 3]"}
\end{Highlighting}
\end{Shaded}

\subsection{Push to Remote}\label{push-to-remote}

\begin{Shaded}
\begin{Highlighting}[]
\FunctionTok{git}\NormalTok{ push origin main}
\end{Highlighting}
\end{Shaded}

\textbf{Git handles symlinks automatically} - they're stored as text
files containing the target path. When someone clones your blog,
symlinks are recreated correctly.

\section{Part 9: Complete Workflow
Summary}\label{part-9-complete-workflow-summary}

\subsection{Timeline and Checklist}\label{timeline-and-checklist}

\begin{longtable}[]{@{}lll@{}}
\toprule\noalign{}
Phase & Time & Task \\
\midrule\noalign{}
\endhead
\bottomrule\noalign{}
\endlastfoot
Assessment & 5 min & Read markdown, plan structure \\
Initialize & 3 min & Create directory, run \texttt{zzcollab} \\
Setup & 1 min & Run \texttt{setup\_symlinks.sh} \\
Convert & 10 min & Adapt markdown → Quarto \\
Media & 5 min & Add images, document sources \\
Metadata & 3 min & Update README, DESCRIPTION \\
Test & 5 min & Verify structure, validate \\
Commit & 2 min & Git add/commit/push \\
\textbf{Total} & \textbf{\textasciitilde40 min} & \textbf{Blog post
ready} \\
\end{longtable}

\subsection{Pre-Publication Checklist}\label{pre-publication-checklist}

\textbf{Content}: - {[} {]} Blog post written (analysis/paper/index.qmd)
- {[} {]} All sections complete - {[} {]} Code examples tested - {[} {]}
Links verified (no 404s) - {[} {]} Grammar/spelling checked

\textbf{Metadata}: - {[} {]} YAML frontmatter correct - {[} {]} Title
descriptive (50-70 chars) - {[} {]} Categories relevant - {[} {]}
Description captures value - {[} {]} \texttt{draft:\ false} (to publish)

\textbf{Media}: - {[} {]} Hero image added - {[} {]} Images documented
in README.md - {[} {]} All image paths relative
(\texttt{media/images/...}) - {[} {]} Image alt text present - {[} {]}
File sizes reasonable (\textless{} 500KB each)

\textbf{Structure}: - {[} {]} Symlinks verified (4 at root) - {[} {]}
README.md customized - {[} {]} DESCRIPTION updated - {[} {]}
\texttt{.gitignore} appropriate

\textbf{Testing}: - {[} {]} Local render successful (if Quarto
installed) - {[} {]} All images display - {[} {]} All links work - {[}
{]} Code syntax highlighted - {[} {]} TOC generates correctly

\section{Part 10: Extending Your Blog
Post}\label{part-10-extending-your-blog-post}

Once published, you can enhance your post:

\subsection{Add R Analysis}\label{add-r-analysis}

Create analysis scripts:

\begin{Shaded}
\begin{Highlighting}[]
\CommentTok{\# analysis/scripts/01\_prepare\_data.R}
\ExtensionTok{library}\ErrorTok{(}\ExtensionTok{tidyverse}\KeywordTok{)}
\CommentTok{\# ... your analysis code ...}
\ExtensionTok{write\_csv}\ErrorTok{(}\ExtensionTok{results,} \StringTok{"analysis/data/derived\_data/results.csv"}\KeywordTok{)}
\end{Highlighting}
\end{Shaded}

Update Makefile:

\begin{Shaded}
\begin{Highlighting}[]
\DecValTok{post{-}analysis:}
\ErrorTok{    }\NormalTok{Rscript analysis/scripts/01\_prepare\_data.R}
\NormalTok{    Rscript analysis/scripts/02\_generate\_figures.R}

\DecValTok{post{-}render:}\DataTypeTok{ post{-}analysis}
\ErrorTok{    }\NormalTok{quarto render index.qmd}
\end{Highlighting}
\end{Shaded}

\subsection{Add Interactive Content}\label{add-interactive-content}

Quarto supports multiple languages:

\begin{Shaded}
\begin{Highlighting}[]
\NormalTok{// Observable JS for interactive visualizations}
\NormalTok{Plot.plot(\{}
\NormalTok{  // ... your D3/Observable code ...}
\NormalTok{\})}
\end{Highlighting}
\end{Shaded}

\begin{Shaded}
\begin{Highlighting}[]
\CommentTok{\# R code with Shiny for interactive elements}
\FunctionTok{library}\NormalTok{(shiny)}
\FunctionTok{library}\NormalTok{(plotly)}
\end{Highlighting}
\end{Shaded}

\subsection{Version Updates}\label{version-updates}

When you update the post:

\begin{Shaded}
\begin{Highlighting}[]
\CommentTok{\# Edit content}
\ExtensionTok{vim}\NormalTok{ analysis/paper/index.qmd}

\CommentTok{\# Update date (if using last{-}modified)}
\CommentTok{\# Or change manually: date: "2025{-}12{-}15"}

\CommentTok{\# Render and test}
\ExtensionTok{quarto}\NormalTok{ render analysis/paper/index.qmd}

\CommentTok{\# Commit update}
\FunctionTok{git}\NormalTok{ add .}
\FunctionTok{git}\NormalTok{ commit }\AttributeTok{{-}m} \StringTok{"Update post: [what changed]"}
\FunctionTok{git}\NormalTok{ push}
\end{Highlighting}
\end{Shaded}

\section{Part 11: Troubleshooting
Guide}\label{part-11-troubleshooting-guide}

\subsection{Symlink Issues}\label{symlink-issues}

\textbf{Symlink appears broken after cloning}:

\begin{Shaded}
\begin{Highlighting}[]
\CommentTok{\# Git may not have recreated symlinks correctly}
\CommentTok{\# Recreate manually:}
\BuiltInTok{cd}\NormalTok{ \textasciitilde{}/prj/qblog/posts/my\_blog\_post}
\FunctionTok{rm}\NormalTok{ index.qmd media figures data}
\FunctionTok{ln} \AttributeTok{{-}s}\NormalTok{ analysis/paper/index.qmd index.qmd}
\FunctionTok{ln} \AttributeTok{{-}s}\NormalTok{ analysis/media media}
\FunctionTok{ln} \AttributeTok{{-}s}\NormalTok{ analysis/figures figures}
\FunctionTok{ln} \AttributeTok{{-}s}\NormalTok{ analysis/data data}
\end{Highlighting}
\end{Shaded}

\subsection{YAML Errors}\label{yaml-errors}

\textbf{``YAML parsing error''}:

\begin{Shaded}
\begin{Highlighting}[]
\CommentTok{\# Check for tabs (they\textquotesingle{}re not allowed)}
\FunctionTok{cat} \AttributeTok{{-}A}\NormalTok{ analysis/paper/index.qmd }\KeywordTok{|} \FunctionTok{head} \AttributeTok{{-}30}
\CommentTok{\# Tabs show as \^{}I, spaces don\textquotesingle{}t}

\CommentTok{\# Fix by using your editor in spaces{-}only mode}
\ExtensionTok{vim}\NormalTok{ analysis/paper/index.qmd}
\ExtensionTok{:set}\NormalTok{ expandtab}
\end{Highlighting}
\end{Shaded}

\subsection{Quarto Issues}\label{quarto-issues}

\textbf{``Quarto executable not found''}:

\begin{Shaded}
\begin{Highlighting}[]
\CommentTok{\# Install Quarto}
\CommentTok{\# macOS:}
\ExtensionTok{brew}\NormalTok{ install quarto}

\CommentTok{\# Or download from: https://quarto.org/docs/get{-}started/}
\end{Highlighting}
\end{Shaded}

\textbf{Code blocks not syntax-highlighted}:

\begin{verbatim}
# Wrong:
```shell
code
\end{verbatim}

\section{Right:}\label{right}

\begin{Shaded}
\begin{Highlighting}[]
\ExtensionTok{code}
\end{Highlighting}
\end{Shaded}

\begin{verbatim}

Quarto recognizes: bash, sh, r, python, sql, javascript, html, css, java, etc.

# Part 12: Best Practices

## Writing for Web Readers

- **Skim-friendly**: Use lots of headings and short paragraphs
- **Scannable**: Bullet points over long paragraphs
- **Linked**: Include links to related docs/resources
- **Examples**: Show actual output, not just descriptions
- **Progressive**: Start simple, build complexity

## Metadata Strategy

**Title**:
- Include searchable keywords
- Be specific (avoid "A Guide To Everything")
- Keep under 70 characters

**Categories**:
- Consistent across your blog
- 2-4 per post
- Enables filtering

**Description**:
- 2-3 sentences maximum
- Appears in blog listing
- Lead with value proposition

## Code Example Quality

**Test code before publishing**:
```bash
# Run the example yourself
bash example-script.sh

# Verify output matches what you document
# Fix any issues before publishing
\end{verbatim}

\textbf{Complete examples}: - Include necessary imports - Show all
required setup - Provide full, working code - Don't make readers guess

\section{Part 13: Real-World Example}\label{part-13-real-world-example}

This blog post itself is a working example! Let me show you:

\subsection{Structure Used Here}\label{structure-used-here}

\begin{verbatim}
markdown_to_blog/
├── index.qmd (symlink) → analysis/paper/index.qmd
├── media/ (symlink) → analysis/media/
├── analysis/
│   ├── paper/
│   │   └── index.qmd (THIS FILE, ~3000 words)
│   └── media/
│       └── images/
│           ├── README.md (document image sources)
│           └── [hero and concept images]
\end{verbatim}

\subsection{What Was Converted}\label{what-was-converted}

Source: \texttt{markdown\_to\_blogpost.md} (process documentation,
\textasciitilde1300 lines)

Converted to: This blog post (teaching audience how to do it)

\subsection{Key Decisions Made}\label{key-decisions-made}

\begin{longtable}[]{@{}ll@{}}
\toprule\noalign{}
Decision & Outcome \\
\midrule\noalign{}
\endhead
\bottomrule\noalign{}
\endlastfoot
\textbf{Slug} & \texttt{markdown\_to\_blog} (kebab-case) \\
\textbf{Profile} & \texttt{ubuntu\_standard\_publishing} \\
\textbf{Content} & Educational (teaching, not documenting) \\
\textbf{Structure} & Progressive (overview → details → checklist) \\
\textbf{Examples} & Reference actual GitHub archive post \\
\end{longtable}

\section{Key Takeaways}\label{key-takeaways}

\begin{tcolorbox}[enhanced jigsaw, colback=white, opacitybacktitle=0.6, toprule=.15mm, colbacktitle=quarto-callout-note-color!10!white, leftrule=.75mm, bottomrule=.15mm, toptitle=1mm, rightrule=.15mm, left=2mm, arc=.35mm, breakable, colframe=quarto-callout-note-color-frame, opacityback=0, coltitle=black, title=\textcolor{quarto-callout-note-color}{\faInfo}\hspace{0.5em}{Summary: From Markdown to Published Blog Post}, bottomtitle=1mm, titlerule=0mm]

\begin{enumerate}
\def\labelenumi{\arabic{enumi}.}
\tightlist
\item
  \textbf{Dual-symlink system} solves the ZZCOLLAB + Quarto
  compatibility puzzle
\item
  \textbf{ZZCOLLAB initialization} gives you reproducibility from the
  start
\item
  \textbf{Minimal conversion effort} - mostly structural (YAML,
  headings)
\item
  \textbf{40 minutes total} - from markdown to committed, ready to
  publish
\item
  \textbf{No parent blog changes} - symlinks make integration seamless
\item
  \textbf{Extensible foundation} - can add analysis, media,
  interactivity later
\end{enumerate}

\textbf{The workflow is repeatable}: Once you've done it once,
converting additional posts becomes faster and more automatic.

\textbf{Most importantly}: Your valuable documentation gets discovered,
reaches your audience, and contributes to your blog's growth.

\end{tcolorbox}

\section{Further Reading}\label{further-reading}

\begin{itemize}
\tightlist
\item
  \href{https://github.com/rgt47/zzcollab}{ZZCOLLAB Framework} - Project
  initialization and management
\item
  \href{https://quarto.org/docs/websites/website-blog.html}{Quarto Blog
  Guide} - Detailed Quarto blog documentation
\item
  \href{https://github.com/benmarwick/rrtools}{rrtools Research
  Compendium} - The convention we follow
\item
  \href{https://quarto.org/docs/reference/formats/html.html}{Quarto YAML
  Reference} - Complete YAML options
\item
  \href{https://git-scm.com/docs/git-ls-files}{Git Symlinks} - How Git
  handles symlinks
\item
  \href{https://quarto.org/docs/tools/jupyter-notebooks.html}{Markdown
  to Quarto} - Quarto format specifics
\end{itemize}




\end{document}
