% Options for packages loaded elsewhere
\PassOptionsToPackage{unicode}{hyperref}
\PassOptionsToPackage{hyphens}{url}
\PassOptionsToPackage{dvipsnames,svgnames,x11names}{xcolor}
%
\documentclass[
  11pt,
  letterpaper,
  DIV=11,
  numbers=noendperiod]{scrartcl}

\usepackage{amsmath,amssymb}
\usepackage{iftex}
\ifPDFTeX
  \usepackage[T1]{fontenc}
  \usepackage[utf8]{inputenc}
  \usepackage{textcomp} % provide euro and other symbols
\else % if luatex or xetex
  \usepackage{unicode-math}
  \defaultfontfeatures{Scale=MatchLowercase}
  \defaultfontfeatures[\rmfamily]{Ligatures=TeX,Scale=1}
\fi
\usepackage{lmodern}
\ifPDFTeX\else  
    % xetex/luatex font selection
\fi
% Use upquote if available, for straight quotes in verbatim environments
\IfFileExists{upquote.sty}{\usepackage{upquote}}{}
\IfFileExists{microtype.sty}{% use microtype if available
  \usepackage[]{microtype}
  \UseMicrotypeSet[protrusion]{basicmath} % disable protrusion for tt fonts
}{}
\makeatletter
\@ifundefined{KOMAClassName}{% if non-KOMA class
  \IfFileExists{parskip.sty}{%
    \usepackage{parskip}
  }{% else
    \setlength{\parindent}{0pt}
    \setlength{\parskip}{6pt plus 2pt minus 1pt}}
}{% if KOMA class
  \KOMAoptions{parskip=half}}
\makeatother
\usepackage{xcolor}
\setlength{\emergencystretch}{3em} % prevent overfull lines
\setcounter{secnumdepth}{5}
% Make \paragraph and \subparagraph free-standing
\makeatletter
\ifx\paragraph\undefined\else
  \let\oldparagraph\paragraph
  \renewcommand{\paragraph}{
    \@ifstar
      \xxxParagraphStar
      \xxxParagraphNoStar
  }
  \newcommand{\xxxParagraphStar}[1]{\oldparagraph*{#1}\mbox{}}
  \newcommand{\xxxParagraphNoStar}[1]{\oldparagraph{#1}\mbox{}}
\fi
\ifx\subparagraph\undefined\else
  \let\oldsubparagraph\subparagraph
  \renewcommand{\subparagraph}{
    \@ifstar
      \xxxSubParagraphStar
      \xxxSubParagraphNoStar
  }
  \newcommand{\xxxSubParagraphStar}[1]{\oldsubparagraph*{#1}\mbox{}}
  \newcommand{\xxxSubParagraphNoStar}[1]{\oldsubparagraph{#1}\mbox{}}
\fi
\makeatother

\usepackage{color}
\usepackage{fancyvrb}
\newcommand{\VerbBar}{|}
\newcommand{\VERB}{\Verb[commandchars=\\\{\}]}
\DefineVerbatimEnvironment{Highlighting}{Verbatim}{commandchars=\\\{\}}
% Add ',fontsize=\small' for more characters per line
\usepackage{framed}
\definecolor{shadecolor}{RGB}{241,243,245}
\newenvironment{Shaded}{\begin{snugshade}}{\end{snugshade}}
\newcommand{\AlertTok}[1]{\textcolor[rgb]{0.68,0.00,0.00}{#1}}
\newcommand{\AnnotationTok}[1]{\textcolor[rgb]{0.37,0.37,0.37}{#1}}
\newcommand{\AttributeTok}[1]{\textcolor[rgb]{0.40,0.45,0.13}{#1}}
\newcommand{\BaseNTok}[1]{\textcolor[rgb]{0.68,0.00,0.00}{#1}}
\newcommand{\BuiltInTok}[1]{\textcolor[rgb]{0.00,0.23,0.31}{#1}}
\newcommand{\CharTok}[1]{\textcolor[rgb]{0.13,0.47,0.30}{#1}}
\newcommand{\CommentTok}[1]{\textcolor[rgb]{0.37,0.37,0.37}{#1}}
\newcommand{\CommentVarTok}[1]{\textcolor[rgb]{0.37,0.37,0.37}{\textit{#1}}}
\newcommand{\ConstantTok}[1]{\textcolor[rgb]{0.56,0.35,0.01}{#1}}
\newcommand{\ControlFlowTok}[1]{\textcolor[rgb]{0.00,0.23,0.31}{\textbf{#1}}}
\newcommand{\DataTypeTok}[1]{\textcolor[rgb]{0.68,0.00,0.00}{#1}}
\newcommand{\DecValTok}[1]{\textcolor[rgb]{0.68,0.00,0.00}{#1}}
\newcommand{\DocumentationTok}[1]{\textcolor[rgb]{0.37,0.37,0.37}{\textit{#1}}}
\newcommand{\ErrorTok}[1]{\textcolor[rgb]{0.68,0.00,0.00}{#1}}
\newcommand{\ExtensionTok}[1]{\textcolor[rgb]{0.00,0.23,0.31}{#1}}
\newcommand{\FloatTok}[1]{\textcolor[rgb]{0.68,0.00,0.00}{#1}}
\newcommand{\FunctionTok}[1]{\textcolor[rgb]{0.28,0.35,0.67}{#1}}
\newcommand{\ImportTok}[1]{\textcolor[rgb]{0.00,0.46,0.62}{#1}}
\newcommand{\InformationTok}[1]{\textcolor[rgb]{0.37,0.37,0.37}{#1}}
\newcommand{\KeywordTok}[1]{\textcolor[rgb]{0.00,0.23,0.31}{\textbf{#1}}}
\newcommand{\NormalTok}[1]{\textcolor[rgb]{0.00,0.23,0.31}{#1}}
\newcommand{\OperatorTok}[1]{\textcolor[rgb]{0.37,0.37,0.37}{#1}}
\newcommand{\OtherTok}[1]{\textcolor[rgb]{0.00,0.23,0.31}{#1}}
\newcommand{\PreprocessorTok}[1]{\textcolor[rgb]{0.68,0.00,0.00}{#1}}
\newcommand{\RegionMarkerTok}[1]{\textcolor[rgb]{0.00,0.23,0.31}{#1}}
\newcommand{\SpecialCharTok}[1]{\textcolor[rgb]{0.37,0.37,0.37}{#1}}
\newcommand{\SpecialStringTok}[1]{\textcolor[rgb]{0.13,0.47,0.30}{#1}}
\newcommand{\StringTok}[1]{\textcolor[rgb]{0.13,0.47,0.30}{#1}}
\newcommand{\VariableTok}[1]{\textcolor[rgb]{0.07,0.07,0.07}{#1}}
\newcommand{\VerbatimStringTok}[1]{\textcolor[rgb]{0.13,0.47,0.30}{#1}}
\newcommand{\WarningTok}[1]{\textcolor[rgb]{0.37,0.37,0.37}{\textit{#1}}}

\providecommand{\tightlist}{%
  \setlength{\itemsep}{0pt}\setlength{\parskip}{0pt}}\usepackage{longtable,booktabs,array}
\usepackage{calc} % for calculating minipage widths
% Correct order of tables after \paragraph or \subparagraph
\usepackage{etoolbox}
\makeatletter
\patchcmd\longtable{\par}{\if@noskipsec\mbox{}\fi\par}{}{}
\makeatother
% Allow footnotes in longtable head/foot
\IfFileExists{footnotehyper.sty}{\usepackage{footnotehyper}}{\usepackage{footnote}}
\makesavenoteenv{longtable}
\usepackage{graphicx}
\makeatletter
\def\maxwidth{\ifdim\Gin@nat@width>\linewidth\linewidth\else\Gin@nat@width\fi}
\def\maxheight{\ifdim\Gin@nat@height>\textheight\textheight\else\Gin@nat@height\fi}
\makeatother
% Scale images if necessary, so that they will not overflow the page
% margins by default, and it is still possible to overwrite the defaults
% using explicit options in \includegraphics[width, height, ...]{}
\setkeys{Gin}{width=\maxwidth,height=\maxheight,keepaspectratio}
% Set default figure placement to htbp
\makeatletter
\def\fps@figure{htbp}
\makeatother

\KOMAoption{captions}{tableheading}
\makeatletter
\@ifpackageloaded{tcolorbox}{}{\usepackage[skins,breakable]{tcolorbox}}
\@ifpackageloaded{fontawesome5}{}{\usepackage{fontawesome5}}
\definecolor{quarto-callout-color}{HTML}{909090}
\definecolor{quarto-callout-note-color}{HTML}{0758E5}
\definecolor{quarto-callout-important-color}{HTML}{CC1914}
\definecolor{quarto-callout-warning-color}{HTML}{EB9113}
\definecolor{quarto-callout-tip-color}{HTML}{00A047}
\definecolor{quarto-callout-caution-color}{HTML}{FC5300}
\definecolor{quarto-callout-color-frame}{HTML}{acacac}
\definecolor{quarto-callout-note-color-frame}{HTML}{4582ec}
\definecolor{quarto-callout-important-color-frame}{HTML}{d9534f}
\definecolor{quarto-callout-warning-color-frame}{HTML}{f0ad4e}
\definecolor{quarto-callout-tip-color-frame}{HTML}{02b875}
\definecolor{quarto-callout-caution-color-frame}{HTML}{fd7e14}
\makeatother
\makeatletter
\@ifpackageloaded{caption}{}{\usepackage{caption}}
\AtBeginDocument{%
\ifdefined\contentsname
  \renewcommand*\contentsname{Table of contents}
\else
  \newcommand\contentsname{Table of contents}
\fi
\ifdefined\listfigurename
  \renewcommand*\listfigurename{List of Figures}
\else
  \newcommand\listfigurename{List of Figures}
\fi
\ifdefined\listtablename
  \renewcommand*\listtablename{List of Tables}
\else
  \newcommand\listtablename{List of Tables}
\fi
\ifdefined\figurename
  \renewcommand*\figurename{Figure}
\else
  \newcommand\figurename{Figure}
\fi
\ifdefined\tablename
  \renewcommand*\tablename{Table}
\else
  \newcommand\tablename{Table}
\fi
}
\@ifpackageloaded{float}{}{\usepackage{float}}
\floatstyle{ruled}
\@ifundefined{c@chapter}{\newfloat{codelisting}{h}{lop}}{\newfloat{codelisting}{h}{lop}[chapter]}
\floatname{codelisting}{Listing}
\newcommand*\listoflistings{\listof{codelisting}{List of Listings}}
\makeatother
\makeatletter
\makeatother
\makeatletter
\@ifpackageloaded{caption}{}{\usepackage{caption}}
\@ifpackageloaded{subcaption}{}{\usepackage{subcaption}}
\makeatother
\ifLuaTeX
  \usepackage{selnolig}  % disable illegal ligatures
\fi
\usepackage{bookmark}

\IfFileExists{xurl.sty}{\usepackage{xurl}}{} % add URL line breaks if available
\urlstyle{same} % disable monospaced font for URLs
\hypersetup{
  pdftitle={Your Engaging Title Here: A Technical Deep Dive},
  pdfauthor={Your Name},
  colorlinks=true,
  linkcolor={blue},
  filecolor={Maroon},
  citecolor={Blue},
  urlcolor={Blue},
  pdfcreator={LaTeX via pandoc}}

\title{Your Engaging Title Here: A Technical Deep Dive}
\usepackage{etoolbox}
\makeatletter
\providecommand{\subtitle}[1]{% add subtitle to \maketitle
  \apptocmd{\@title}{\par {\large #1 \par}}{}{}
}
\makeatother
\subtitle{A compelling subtitle that expands on the main title and hooks
the reader}
\author{Your Name}
\date{2025-01-01}

\begin{document}
\maketitle

\begin{figure}[H]

{\centering \includegraphics{../../images/posts/penguins-library-7755210_1280.jpg}

}

\caption{Engaging hero image that introduces your topic visually}

\end{figure}%

\emph{Photo caption with attribution if needed. This image sets the
visual tone for your entire post.}

\section{Introduction}\label{introduction}

Welcome to this comprehensive exploration of {[}topic{]}! In this post,
we'll journey through {[}brief overview of content{]}. This topic is
particularly relevant for {[}target audience{]} because
{[}motivation{]}.

{[}Problem statement or context paragraph. Example: ``Data scientists
often struggle with X, which leads to Y problems. Understanding Z is
crucial for\ldots{}''{]}

In this post, we'll focus on:

\begin{itemize}
\tightlist
\item
  {[}Learning objective 1: specific, actionable{]}
\item
  {[}Learning objective 2: builds on previous{]}
\item
  {[}Learning objective 3: practical application{]}
\item
  {[}Learning objective 4: advanced concept{]}
\end{itemize}

By the end of this post, you'll have a solid understanding of {[}main
takeaway{]} and be able to {[}practical skill{]}.

\section{Prerequisites and Setup}\label{prerequisites-and-setup}

Before we begin, let's ensure we have the right tools:

\textbf{Required Packages:}

\begin{Shaded}
\begin{Highlighting}[]
\CommentTok{\# Install required packages if not already installed}
\FunctionTok{install.packages}\NormalTok{(}\FunctionTok{c}\NormalTok{(}\StringTok{"tidyverse"}\NormalTok{, }\StringTok{"broom"}\NormalTok{, }\StringTok{"knitr"}\NormalTok{, }\StringTok{"patchwork"}\NormalTok{))}
\end{Highlighting}
\end{Shaded}

\textbf{Load Libraries:}

\begin{Shaded}
\begin{Highlighting}[]
\FunctionTok{library}\NormalTok{(tidyverse)}
\FunctionTok{library}\NormalTok{(broom)}
\FunctionTok{library}\NormalTok{(knitr)}
\FunctionTok{library}\NormalTok{(patchwork)}

\CommentTok{\# Set theme for consistent plotting}
\FunctionTok{theme\_set}\NormalTok{(}\FunctionTok{theme\_minimal}\NormalTok{(}\AttributeTok{base\_size =} \DecValTok{12}\NormalTok{))}

\CommentTok{\# Set custom colors (adjust to your preference)}
\NormalTok{custom\_colors }\OtherTok{\textless{}{-}} \FunctionTok{c}\NormalTok{(}\StringTok{"\#FF6B6B"}\NormalTok{, }\StringTok{"\#4ECDC4"}\NormalTok{, }\StringTok{"\#45B7D1"}\NormalTok{, }\StringTok{"\#96CEB4"}\NormalTok{)}
\end{Highlighting}
\end{Shaded}

\textbf{Background Knowledge:} - Basic familiarity with R and ggplot2 -
Understanding of {[}prerequisite concept 1{]} - Optional: Experience
with {[}advanced prerequisite{]}

\section{Section 1: Data Overview and Initial
Exploration}\label{section-1-data-overview-and-initial-exploration}

Let's start by getting acquainted with our dataset:

\begin{Shaded}
\begin{Highlighting}[]
\CommentTok{\# Load the mtcars dataset}
\FunctionTok{data}\NormalTok{(mtcars)}

\CommentTok{\# Basic dataset information}
\FunctionTok{cat}\NormalTok{(}\StringTok{"📊 Dataset Overview}\SpecialCharTok{\textbackslash{}n}\StringTok{"}\NormalTok{)}
\end{Highlighting}
\end{Shaded}

\begin{verbatim}
📊 Dataset Overview
\end{verbatim}

\begin{Shaded}
\begin{Highlighting}[]
\FunctionTok{cat}\NormalTok{(}\StringTok{"==================}\SpecialCharTok{\textbackslash{}n}\StringTok{"}\NormalTok{)}
\end{Highlighting}
\end{Shaded}

\begin{verbatim}
==================
\end{verbatim}

\begin{Shaded}
\begin{Highlighting}[]
\FunctionTok{cat}\NormalTok{(}\StringTok{"Dimensions:"}\NormalTok{, }\FunctionTok{nrow}\NormalTok{(mtcars), }\StringTok{"observations ×"}\NormalTok{, }\FunctionTok{ncol}\NormalTok{(mtcars), }\StringTok{"variables}\SpecialCharTok{\textbackslash{}n\textbackslash{}n}\StringTok{"}\NormalTok{)}
\end{Highlighting}
\end{Shaded}

\begin{verbatim}
Dimensions: 32 observations × 11 variables
\end{verbatim}

\begin{Shaded}
\begin{Highlighting}[]
\CommentTok{\# Display structure}
\FunctionTok{glimpse}\NormalTok{(mtcars)}
\end{Highlighting}
\end{Shaded}

\begin{verbatim}
Rows: 32
Columns: 11
$ mpg  <dbl> 21.0, 21.0, 22.8, 21.4, 18.7, 18.1, 14.3, 24.4, 22.8, 19.2, 17.8,~
$ cyl  <dbl> 6, 6, 4, 6, 8, 6, 8, 4, 4, 6, 6, 8, 8, 8, 8, 8, 8, 4, 4, 4, 4, 8,~
$ disp <dbl> 160.0, 160.0, 108.0, 258.0, 360.0, 225.0, 360.0, 146.7, 140.8, 16~
$ hp   <dbl> 110, 110, 93, 110, 175, 105, 245, 62, 95, 123, 123, 180, 180, 180~
$ drat <dbl> 3.90, 3.90, 3.85, 3.08, 3.15, 2.76, 3.21, 3.69, 3.92, 3.92, 3.92,~
$ wt   <dbl> 2.620, 2.875, 2.320, 3.215, 3.440, 3.460, 3.570, 3.190, 3.150, 3.~
$ qsec <dbl> 16.46, 17.02, 18.61, 19.44, 17.02, 20.22, 15.84, 20.00, 22.90, 18~
$ vs   <dbl> 0, 0, 1, 1, 0, 1, 0, 1, 1, 1, 1, 0, 0, 0, 0, 0, 0, 1, 1, 1, 1, 0,~
$ am   <dbl> 1, 1, 1, 0, 0, 0, 0, 0, 0, 0, 0, 0, 0, 0, 0, 0, 0, 1, 1, 1, 0, 0,~
$ gear <dbl> 4, 4, 4, 3, 3, 3, 3, 4, 4, 4, 4, 3, 3, 3, 3, 3, 3, 4, 4, 4, 3, 3,~
$ carb <dbl> 4, 4, 1, 1, 2, 1, 4, 2, 2, 4, 4, 3, 3, 3, 4, 4, 4, 1, 2, 1, 1, 2,~
\end{verbatim}

\subsection{Understanding the
Variables}\label{understanding-the-variables}

The mtcars dataset contains the following key variables:

\begin{Shaded}
\begin{Highlighting}[]
\CommentTok{\# Summary statistics}
\NormalTok{summary\_table }\OtherTok{\textless{}{-}}\NormalTok{ mtcars }\SpecialCharTok{\%\textgreater{}\%}
  \FunctionTok{summarise}\NormalTok{(}
    \AttributeTok{n =} \FunctionTok{n}\NormalTok{(),}
    \AttributeTok{mpg\_mean =} \FunctionTok{round}\NormalTok{(}\FunctionTok{mean}\NormalTok{(mpg), }\DecValTok{1}\NormalTok{),}
    \AttributeTok{mpg\_sd =} \FunctionTok{round}\NormalTok{(}\FunctionTok{sd}\NormalTok{(mpg), }\DecValTok{1}\NormalTok{),}
    \AttributeTok{hp\_mean =} \FunctionTok{round}\NormalTok{(}\FunctionTok{mean}\NormalTok{(hp), }\DecValTok{0}\NormalTok{),}
    \AttributeTok{hp\_sd =} \FunctionTok{round}\NormalTok{(}\FunctionTok{sd}\NormalTok{(hp), }\DecValTok{0}\NormalTok{)}
\NormalTok{  )}

\FunctionTok{kable}\NormalTok{(summary\_table,}
      \AttributeTok{caption =} \StringTok{"Summary Statistics for Key Variables"}\NormalTok{,}
      \AttributeTok{col.names =} \FunctionTok{c}\NormalTok{(}\StringTok{"N"}\NormalTok{, }\StringTok{"MPG Mean"}\NormalTok{, }\StringTok{"MPG SD"}\NormalTok{, }\StringTok{"HP Mean"}\NormalTok{, }\StringTok{"HP SD"}\NormalTok{))}
\end{Highlighting}
\end{Shaded}

\begin{longtable}[]{@{}rrrrr@{}}
\caption{Summary Statistics for Key Variables}\tabularnewline
\toprule\noalign{}
N & MPG Mean & MPG SD & HP Mean & HP SD \\
\midrule\noalign{}
\endfirsthead
\toprule\noalign{}
N & MPG Mean & MPG SD & HP Mean & HP SD \\
\midrule\noalign{}
\endhead
\bottomrule\noalign{}
\endlastfoot
32 & 20.1 & 6 & 147 & 69 \\
\end{longtable}

\section{Section 2: Exploratory Data
Analysis}\label{section-2-exploratory-data-analysis}

Let's explore the relationships between variables:

\begin{Shaded}
\begin{Highlighting}[]
\CommentTok{\# Create distribution plots}
\NormalTok{p1 }\OtherTok{\textless{}{-}} \FunctionTok{ggplot}\NormalTok{(mtcars, }\FunctionTok{aes}\NormalTok{(}\AttributeTok{x =}\NormalTok{ mpg)) }\SpecialCharTok{+}
  \FunctionTok{geom\_histogram}\NormalTok{(}\AttributeTok{bins =} \DecValTok{15}\NormalTok{, }\AttributeTok{fill =}\NormalTok{ custom\_colors[}\DecValTok{1}\NormalTok{], }\AttributeTok{alpha =} \FloatTok{0.7}\NormalTok{) }\SpecialCharTok{+}
  \FunctionTok{labs}\NormalTok{(}\AttributeTok{title =} \StringTok{"Distribution of MPG"}\NormalTok{, }\AttributeTok{x =} \StringTok{"Miles Per Gallon"}\NormalTok{, }\AttributeTok{y =} \StringTok{"Count"}\NormalTok{) }\SpecialCharTok{+}
  \FunctionTok{theme\_minimal}\NormalTok{()}

\NormalTok{p2 }\OtherTok{\textless{}{-}} \FunctionTok{ggplot}\NormalTok{(mtcars, }\FunctionTok{aes}\NormalTok{(}\AttributeTok{x =} \FunctionTok{factor}\NormalTok{(cyl), }\AttributeTok{y =}\NormalTok{ mpg, }\AttributeTok{fill =} \FunctionTok{factor}\NormalTok{(cyl))) }\SpecialCharTok{+}
  \FunctionTok{geom\_boxplot}\NormalTok{(}\AttributeTok{alpha =} \FloatTok{0.7}\NormalTok{) }\SpecialCharTok{+}
  \FunctionTok{scale\_fill\_manual}\NormalTok{(}\AttributeTok{values =}\NormalTok{ custom\_colors) }\SpecialCharTok{+}
  \FunctionTok{labs}\NormalTok{(}\AttributeTok{title =} \StringTok{"MPG by Cylinder Count"}\NormalTok{,}
       \AttributeTok{x =} \StringTok{"Number of Cylinders"}\NormalTok{, }\AttributeTok{y =} \StringTok{"Miles Per Gallon"}\NormalTok{) }\SpecialCharTok{+}
  \FunctionTok{theme\_minimal}\NormalTok{() }\SpecialCharTok{+}
  \FunctionTok{theme}\NormalTok{(}\AttributeTok{legend.position =} \StringTok{"none"}\NormalTok{)}

\CommentTok{\# Combine plots}
\NormalTok{combined\_plot }\OtherTok{\textless{}{-}}\NormalTok{ p1 }\SpecialCharTok{+}\NormalTok{ p2}
\FunctionTok{print}\NormalTok{(combined\_plot)}
\end{Highlighting}
\end{Shaded}

\includegraphics{BLOG_POST_TEMPLATE_V2_files/figure-pdf/unnamed-chunk-5-1.pdf}

\begin{Shaded}
\begin{Highlighting}[]
\CommentTok{\# Save the plot}
\FunctionTok{ggsave}\NormalTok{(}\StringTok{"eda{-}overview.png"}\NormalTok{, }\AttributeTok{plot =}\NormalTok{ combined\_plot, }\AttributeTok{width =} \DecValTok{10}\NormalTok{, }\AttributeTok{height =} \DecValTok{5}\NormalTok{, }\AttributeTok{dpi =} \DecValTok{300}\NormalTok{)}
\end{Highlighting}
\end{Shaded}

\begin{figure}[H]

{\centering \includegraphics{eda-overview.png}

}

\caption{Overview of fuel efficiency distributions showing variation
across cylinder counts}

\end{figure}%

\includegraphics[width=0.5\textwidth,height=\textheight]{../../images/posts/penguins-26046_1280.jpg}
\emph{``Taking a closer look at the patterns in our data\ldots{}''}

\subsection{Correlation Analysis}\label{correlation-analysis}

\begin{Shaded}
\begin{Highlighting}[]
\CommentTok{\# Calculate correlations}
\NormalTok{correlations }\OtherTok{\textless{}{-}} \FunctionTok{cor}\NormalTok{(mtcars) }\SpecialCharTok{\%\textgreater{}\%}
  \FunctionTok{as.data.frame}\NormalTok{() }\SpecialCharTok{\%\textgreater{}\%}
  \FunctionTok{rownames\_to\_column}\NormalTok{(}\StringTok{"var1"}\NormalTok{) }\SpecialCharTok{\%\textgreater{}\%}
  \FunctionTok{pivot\_longer}\NormalTok{(}\SpecialCharTok{{-}}\NormalTok{var1, }\AttributeTok{names\_to =} \StringTok{"var2"}\NormalTok{, }\AttributeTok{values\_to =} \StringTok{"correlation"}\NormalTok{) }\SpecialCharTok{\%\textgreater{}\%}
  \FunctionTok{filter}\NormalTok{(var1 }\SpecialCharTok{==} \StringTok{"mpg"}\NormalTok{, var2 }\SpecialCharTok{!=} \StringTok{"mpg"}\NormalTok{) }\SpecialCharTok{\%\textgreater{}\%}
  \FunctionTok{arrange}\NormalTok{(}\FunctionTok{desc}\NormalTok{(}\FunctionTok{abs}\NormalTok{(correlation)))}

\CommentTok{\# Display top correlations}
\FunctionTok{cat}\NormalTok{(}\StringTok{"🔍 Strongest Correlations with MPG:}\SpecialCharTok{\textbackslash{}n}\StringTok{"}\NormalTok{)}
\end{Highlighting}
\end{Shaded}

\begin{verbatim}
🔍 Strongest Correlations with MPG:
\end{verbatim}

\begin{Shaded}
\begin{Highlighting}[]
\FunctionTok{print}\NormalTok{(correlations }\SpecialCharTok{\%\textgreater{}\%} \FunctionTok{head}\NormalTok{(}\DecValTok{5}\NormalTok{), }\AttributeTok{n =} \DecValTok{5}\NormalTok{)}
\end{Highlighting}
\end{Shaded}

\begin{verbatim}
# A tibble: 5 x 3
  var1  var2  correlation
  <chr> <chr>       <dbl>
1 mpg   wt         -0.868
2 mpg   cyl        -0.852
3 mpg   disp       -0.848
4 mpg   hp         -0.776
5 mpg   drat        0.681
\end{verbatim}

\begin{Shaded}
\begin{Highlighting}[]
\CommentTok{\# Visualize key relationship}
\NormalTok{key\_plot }\OtherTok{\textless{}{-}} \FunctionTok{ggplot}\NormalTok{(mtcars, }\FunctionTok{aes}\NormalTok{(}\AttributeTok{x =}\NormalTok{ wt, }\AttributeTok{y =}\NormalTok{ mpg, }\AttributeTok{color =} \FunctionTok{factor}\NormalTok{(cyl))) }\SpecialCharTok{+}
  \FunctionTok{geom\_point}\NormalTok{(}\AttributeTok{size =} \DecValTok{3}\NormalTok{, }\AttributeTok{alpha =} \FloatTok{0.7}\NormalTok{) }\SpecialCharTok{+}
  \FunctionTok{geom\_smooth}\NormalTok{(}\AttributeTok{method =} \StringTok{"lm"}\NormalTok{, }\AttributeTok{se =} \ConstantTok{FALSE}\NormalTok{, }\AttributeTok{color =} \StringTok{"black"}\NormalTok{, }\AttributeTok{linetype =} \StringTok{"dashed"}\NormalTok{) }\SpecialCharTok{+}
  \FunctionTok{scale\_color\_manual}\NormalTok{(}\AttributeTok{values =}\NormalTok{ custom\_colors, }\AttributeTok{name =} \StringTok{"Cylinders"}\NormalTok{) }\SpecialCharTok{+}
  \FunctionTok{labs}\NormalTok{(}\AttributeTok{title =} \StringTok{"Relationship Between Weight and Fuel Efficiency"}\NormalTok{,}
       \AttributeTok{x =} \StringTok{"Weight (1000 lbs)"}\NormalTok{, }\AttributeTok{y =} \StringTok{"Miles Per Gallon"}\NormalTok{) }\SpecialCharTok{+}
  \FunctionTok{theme\_minimal}\NormalTok{()}

\FunctionTok{print}\NormalTok{(key\_plot)}
\end{Highlighting}
\end{Shaded}

\includegraphics{BLOG_POST_TEMPLATE_V2_files/figure-pdf/unnamed-chunk-6-1.pdf}

\begin{Shaded}
\begin{Highlighting}[]
\FunctionTok{ggsave}\NormalTok{(}\StringTok{"correlation{-}plot.png"}\NormalTok{, }\AttributeTok{plot =}\NormalTok{ key\_plot, }\AttributeTok{width =} \DecValTok{8}\NormalTok{, }\AttributeTok{height =} \DecValTok{5}\NormalTok{, }\AttributeTok{dpi =} \DecValTok{300}\NormalTok{)}
\end{Highlighting}
\end{Shaded}

\begin{figure}[H]

{\centering \includegraphics{correlation-plot.png}

}

\caption{Scatter plot showing negative relationship between vehicle
weight and fuel efficiency}

\end{figure}%

\section{Section 3: Statistical
Modeling}\label{section-3-statistical-modeling}

Now let's build a statistical model to understand these relationships:

\includegraphics[width=0.45\textwidth,height=\textheight]{../../images/posts/penguin-gentoo-penguin-7073394_1280.jpg}
\emph{``Building our statistical model\ldots{}''}

\subsection{Simple Linear Regression}\label{simple-linear-regression}

\begin{Shaded}
\begin{Highlighting}[]
\CommentTok{\# Fit simple linear model}
\NormalTok{simple\_model }\OtherTok{\textless{}{-}} \FunctionTok{lm}\NormalTok{(mpg }\SpecialCharTok{\textasciitilde{}}\NormalTok{ wt, }\AttributeTok{data =}\NormalTok{ mtcars)}

\CommentTok{\# Extract model information with confidence intervals}
\NormalTok{model\_summary }\OtherTok{\textless{}{-}} \FunctionTok{tidy}\NormalTok{(simple\_model, }\AttributeTok{conf.int =} \ConstantTok{TRUE}\NormalTok{)}
\NormalTok{model\_metrics }\OtherTok{\textless{}{-}} \FunctionTok{glance}\NormalTok{(simple\_model)}

\CommentTok{\# Display results}
\FunctionTok{cat}\NormalTok{(}\StringTok{"📊 Simple Linear Model Results:}\SpecialCharTok{\textbackslash{}n}\StringTok{"}\NormalTok{)}
\end{Highlighting}
\end{Shaded}

\begin{verbatim}
📊 Simple Linear Model Results:
\end{verbatim}

\begin{Shaded}
\begin{Highlighting}[]
\FunctionTok{cat}\NormalTok{(}\StringTok{"================================}\SpecialCharTok{\textbackslash{}n}\StringTok{"}\NormalTok{)}
\end{Highlighting}
\end{Shaded}

\begin{verbatim}
================================
\end{verbatim}

\begin{Shaded}
\begin{Highlighting}[]
\FunctionTok{cat}\NormalTok{(}\FunctionTok{sprintf}\NormalTok{(}\StringTok{"R{-}squared: \%.3f (\%.1f\%\% of variance explained)}\SpecialCharTok{\textbackslash{}n}\StringTok{"}\NormalTok{,}
\NormalTok{            model\_metrics}\SpecialCharTok{$}\NormalTok{r.squared, model\_metrics}\SpecialCharTok{$}\NormalTok{r.squared }\SpecialCharTok{*} \DecValTok{100}\NormalTok{))}
\end{Highlighting}
\end{Shaded}

\begin{verbatim}
R-squared: 0.753 (75.3% of variance explained)
\end{verbatim}

\begin{Shaded}
\begin{Highlighting}[]
\FunctionTok{cat}\NormalTok{(}\FunctionTok{sprintf}\NormalTok{(}\StringTok{"RMSE: \%.2f MPG}\SpecialCharTok{\textbackslash{}n}\StringTok{"}\NormalTok{, }\FunctionTok{sigma}\NormalTok{(simple\_model)))}
\end{Highlighting}
\end{Shaded}

\begin{verbatim}
RMSE: 3.05 MPG
\end{verbatim}

\begin{Shaded}
\begin{Highlighting}[]
\FunctionTok{cat}\NormalTok{(}\FunctionTok{sprintf}\NormalTok{(}\StringTok{"F{-}statistic: \%.1f (p \textless{} 0.001)}\SpecialCharTok{\textbackslash{}n\textbackslash{}n}\StringTok{"}\NormalTok{, model\_metrics}\SpecialCharTok{$}\NormalTok{statistic))}
\end{Highlighting}
\end{Shaded}

\begin{verbatim}
F-statistic: 91.4 (p < 0.001)
\end{verbatim}

\begin{Shaded}
\begin{Highlighting}[]
\CommentTok{\# Model equation}
\FunctionTok{cat}\NormalTok{(}\StringTok{"🧮 Model Equation:}\SpecialCharTok{\textbackslash{}n}\StringTok{"}\NormalTok{)}
\end{Highlighting}
\end{Shaded}

\begin{verbatim}
🧮 Model Equation:
\end{verbatim}

\begin{Shaded}
\begin{Highlighting}[]
\FunctionTok{cat}\NormalTok{(}\FunctionTok{sprintf}\NormalTok{(}\StringTok{"MPG = \%.2f + \%.2f × Weight}\SpecialCharTok{\textbackslash{}n}\StringTok{"}\NormalTok{,}
\NormalTok{            model\_summary}\SpecialCharTok{$}\NormalTok{estimate[}\DecValTok{1}\NormalTok{], model\_summary}\SpecialCharTok{$}\NormalTok{estimate[}\DecValTok{2}\NormalTok{]))}
\end{Highlighting}
\end{Shaded}

\begin{verbatim}
MPG = 37.29 + -5.34 × Weight
\end{verbatim}

\begin{Shaded}
\begin{Highlighting}[]
\FunctionTok{cat}\NormalTok{(}\FunctionTok{sprintf}\NormalTok{(}\StringTok{"Slope 95\%\% CI: [\%.2f, \%.2f]}\SpecialCharTok{\textbackslash{}n}\StringTok{"}\NormalTok{,}
\NormalTok{            model\_summary}\SpecialCharTok{$}\NormalTok{conf.low[}\DecValTok{2}\NormalTok{], model\_summary}\SpecialCharTok{$}\NormalTok{conf.high[}\DecValTok{2}\NormalTok{]))}
\end{Highlighting}
\end{Shaded}

\begin{verbatim}
Slope 95% CI: [-6.49, -4.20]
\end{verbatim}

\begin{Shaded}
\begin{Highlighting}[]
\CommentTok{\# Generate predictions}
\NormalTok{new\_data }\OtherTok{\textless{}{-}} \FunctionTok{tibble}\NormalTok{(}\AttributeTok{wt =} \FunctionTok{c}\NormalTok{(}\DecValTok{2}\NormalTok{, }\DecValTok{3}\NormalTok{, }\DecValTok{4}\NormalTok{))}
\NormalTok{predictions }\OtherTok{\textless{}{-}} \FunctionTok{predict}\NormalTok{(simple\_model, }\AttributeTok{newdata =}\NormalTok{ new\_data, }\AttributeTok{interval =} \StringTok{"confidence"}\NormalTok{)}

\FunctionTok{cat}\NormalTok{(}\StringTok{"}\SpecialCharTok{\textbackslash{}n}\StringTok{📝 Example Predictions (95\% CI):}\SpecialCharTok{\textbackslash{}n}\StringTok{"}\NormalTok{)}
\end{Highlighting}
\end{Shaded}

\begin{verbatim}

📝 Example Predictions (95% CI):
\end{verbatim}

\begin{Shaded}
\begin{Highlighting}[]
\ControlFlowTok{for}\NormalTok{(i }\ControlFlowTok{in} \DecValTok{1}\SpecialCharTok{:}\FunctionTok{nrow}\NormalTok{(new\_data)) \{}
  \FunctionTok{cat}\NormalTok{(}\FunctionTok{sprintf}\NormalTok{(}\StringTok{"• \%.1f thousand lbs: \%.1f MPG [\%.1f, \%.1f]}\SpecialCharTok{\textbackslash{}n}\StringTok{"}\NormalTok{,}
\NormalTok{              new\_data}\SpecialCharTok{$}\NormalTok{wt[i],}
\NormalTok{              predictions[i, }\StringTok{"fit"}\NormalTok{],}
\NormalTok{              predictions[i, }\StringTok{"lwr"}\NormalTok{],}
\NormalTok{              predictions[i, }\StringTok{"upr"}\NormalTok{]))}
\NormalTok{\}}
\end{Highlighting}
\end{Shaded}

\begin{verbatim}
• 2.0 thousand lbs: 26.6 MPG [24.8, 28.4]
• 3.0 thousand lbs: 21.3 MPG [20.1, 22.4]
• 4.0 thousand lbs: 15.9 MPG [14.5, 17.3]
\end{verbatim}

\subsection{Model Visualization}\label{model-visualization}

\begin{Shaded}
\begin{Highlighting}[]
\CommentTok{\# Visualize model fit with confidence bands}
\NormalTok{model\_plot }\OtherTok{\textless{}{-}} \FunctionTok{ggplot}\NormalTok{(mtcars, }\FunctionTok{aes}\NormalTok{(}\AttributeTok{x =}\NormalTok{ wt, }\AttributeTok{y =}\NormalTok{ mpg)) }\SpecialCharTok{+}
  \FunctionTok{geom\_point}\NormalTok{(}\FunctionTok{aes}\NormalTok{(}\AttributeTok{color =} \FunctionTok{factor}\NormalTok{(cyl)), }\AttributeTok{size =} \DecValTok{3}\NormalTok{, }\AttributeTok{alpha =} \FloatTok{0.6}\NormalTok{) }\SpecialCharTok{+}
  \FunctionTok{geom\_smooth}\NormalTok{(}\AttributeTok{method =} \StringTok{"lm"}\NormalTok{, }\AttributeTok{color =} \StringTok{"black"}\NormalTok{, }\AttributeTok{fill =} \StringTok{"gray80"}\NormalTok{) }\SpecialCharTok{+}
  \FunctionTok{scale\_color\_manual}\NormalTok{(}\AttributeTok{values =}\NormalTok{ custom\_colors, }\AttributeTok{name =} \StringTok{"Cylinders"}\NormalTok{) }\SpecialCharTok{+}
  \FunctionTok{labs}\NormalTok{(}\AttributeTok{title =} \StringTok{"Linear Model: MPG \textasciitilde{} Weight"}\NormalTok{,}
       \AttributeTok{subtitle =} \StringTok{"Gray band shows 95\% confidence interval"}\NormalTok{,}
       \AttributeTok{x =} \StringTok{"Weight (1000 lbs)"}\NormalTok{, }\AttributeTok{y =} \StringTok{"Miles Per Gallon"}\NormalTok{) }\SpecialCharTok{+}
  \FunctionTok{theme\_minimal}\NormalTok{()}

\FunctionTok{print}\NormalTok{(model\_plot)}
\end{Highlighting}
\end{Shaded}

\includegraphics{BLOG_POST_TEMPLATE_V2_files/figure-pdf/unnamed-chunk-8-1.pdf}

\begin{Shaded}
\begin{Highlighting}[]
\FunctionTok{ggsave}\NormalTok{(}\StringTok{"model{-}plot.png"}\NormalTok{, }\AttributeTok{plot =}\NormalTok{ model\_plot, }\AttributeTok{width =} \DecValTok{8}\NormalTok{, }\AttributeTok{height =} \DecValTok{5}\NormalTok{, }\AttributeTok{dpi =} \DecValTok{300}\NormalTok{)}
\end{Highlighting}
\end{Shaded}

\begin{figure}[H]

{\centering \includegraphics{model-plot.png}

}

\caption{Linear regression model showing relationship between weight and
fuel efficiency with confidence bands}

\end{figure}%

\section{Section 4: Model Diagnostics and
Validation}\label{section-4-model-diagnostics-and-validation}

\includegraphics[width=0.4\textwidth,height=\textheight]{../../images/posts/penguins-7553626_1280.jpg}
\emph{``Always validate your assumptions!''}

\subsection{Checking Model
Assumptions}\label{checking-model-assumptions}

Before trusting our results, we need to validate key assumptions:

\begin{Shaded}
\begin{Highlighting}[]
\CommentTok{\# Add diagnostic information}
\NormalTok{mtcars\_diagnostics }\OtherTok{\textless{}{-}}\NormalTok{ mtcars }\SpecialCharTok{\%\textgreater{}\%}
  \FunctionTok{mutate}\NormalTok{(}
    \AttributeTok{predicted =} \FunctionTok{predict}\NormalTok{(simple\_model),}
    \AttributeTok{residuals =} \FunctionTok{residuals}\NormalTok{(simple\_model),}
    \AttributeTok{standardized\_residuals =} \FunctionTok{rstandard}\NormalTok{(simple\_model)}
\NormalTok{  )}

\CommentTok{\# Check for outliers}
\NormalTok{outliers }\OtherTok{\textless{}{-}} \FunctionTok{which}\NormalTok{(}\FunctionTok{abs}\NormalTok{(mtcars\_diagnostics}\SpecialCharTok{$}\NormalTok{standardized\_residuals) }\SpecialCharTok{\textgreater{}} \FloatTok{2.5}\NormalTok{)}

\FunctionTok{cat}\NormalTok{(}\StringTok{"⚠️  Model Diagnostic Checks:}\SpecialCharTok{\textbackslash{}n}\StringTok{"}\NormalTok{)}
\end{Highlighting}
\end{Shaded}

\begin{verbatim}
⚠️  Model Diagnostic Checks:
\end{verbatim}

\begin{Shaded}
\begin{Highlighting}[]
\FunctionTok{cat}\NormalTok{(}\StringTok{"============================}\SpecialCharTok{\textbackslash{}n}\StringTok{"}\NormalTok{)}
\end{Highlighting}
\end{Shaded}

\begin{verbatim}
============================
\end{verbatim}

\begin{Shaded}
\begin{Highlighting}[]
\FunctionTok{cat}\NormalTok{(}\FunctionTok{sprintf}\NormalTok{(}\StringTok{"• Potential outliers: \%d observations (\textgreater{}2.5 SD)}\SpecialCharTok{\textbackslash{}n}\StringTok{"}\NormalTok{, }\FunctionTok{length}\NormalTok{(outliers)))}
\end{Highlighting}
\end{Shaded}

\begin{verbatim}
• Potential outliers: 0 observations (>2.5 SD)
\end{verbatim}

\begin{Shaded}
\begin{Highlighting}[]
\FunctionTok{cat}\NormalTok{(}\FunctionTok{sprintf}\NormalTok{(}\StringTok{"• Residual standard error: \%.2f MPG}\SpecialCharTok{\textbackslash{}n}\StringTok{"}\NormalTok{, }\FunctionTok{sigma}\NormalTok{(simple\_model)))}
\end{Highlighting}
\end{Shaded}

\begin{verbatim}
• Residual standard error: 3.05 MPG
\end{verbatim}

\begin{Shaded}
\begin{Highlighting}[]
\CommentTok{\# Create diagnostic plots}
\NormalTok{diag\_plot }\OtherTok{\textless{}{-}} \FunctionTok{ggplot}\NormalTok{(mtcars\_diagnostics, }\FunctionTok{aes}\NormalTok{(}\AttributeTok{x =}\NormalTok{ predicted, }\AttributeTok{y =}\NormalTok{ standardized\_residuals)) }\SpecialCharTok{+}
  \FunctionTok{geom\_point}\NormalTok{(}\FunctionTok{aes}\NormalTok{(}\AttributeTok{color =} \FunctionTok{factor}\NormalTok{(cyl)), }\AttributeTok{size =} \DecValTok{3}\NormalTok{, }\AttributeTok{alpha =} \FloatTok{0.6}\NormalTok{) }\SpecialCharTok{+}
  \FunctionTok{geom\_hline}\NormalTok{(}\AttributeTok{yintercept =} \FunctionTok{c}\NormalTok{(}\SpecialCharTok{{-}}\DecValTok{2}\NormalTok{, }\DecValTok{0}\NormalTok{, }\DecValTok{2}\NormalTok{),}
             \AttributeTok{linetype =} \FunctionTok{c}\NormalTok{(}\StringTok{"dashed"}\NormalTok{, }\StringTok{"solid"}\NormalTok{, }\StringTok{"dashed"}\NormalTok{),}
             \AttributeTok{color =} \FunctionTok{c}\NormalTok{(}\StringTok{"red"}\NormalTok{, }\StringTok{"black"}\NormalTok{, }\StringTok{"red"}\NormalTok{)) }\SpecialCharTok{+}
  \FunctionTok{scale\_color\_manual}\NormalTok{(}\AttributeTok{values =}\NormalTok{ custom\_colors, }\AttributeTok{name =} \StringTok{"Cylinders"}\NormalTok{) }\SpecialCharTok{+}
  \FunctionTok{labs}\NormalTok{(}\AttributeTok{title =} \StringTok{"Residual Diagnostic Plot"}\NormalTok{,}
       \AttributeTok{subtitle =} \StringTok{"Checking for patterns in model errors"}\NormalTok{,}
       \AttributeTok{x =} \StringTok{"Predicted MPG"}\NormalTok{, }\AttributeTok{y =} \StringTok{"Standardized Residuals"}\NormalTok{) }\SpecialCharTok{+}
  \FunctionTok{theme\_minimal}\NormalTok{()}

\FunctionTok{print}\NormalTok{(diag\_plot)}
\end{Highlighting}
\end{Shaded}

\includegraphics{BLOG_POST_TEMPLATE_V2_files/figure-pdf/unnamed-chunk-9-1.pdf}

\begin{Shaded}
\begin{Highlighting}[]
\FunctionTok{ggsave}\NormalTok{(}\StringTok{"diagnostics{-}plot.png"}\NormalTok{, }\AttributeTok{plot =}\NormalTok{ diag\_plot, }\AttributeTok{width =} \DecValTok{8}\NormalTok{, }\AttributeTok{height =} \DecValTok{5}\NormalTok{, }\AttributeTok{dpi =} \DecValTok{300}\NormalTok{)}
\end{Highlighting}
\end{Shaded}

\begin{figure}[H]

{\centering \includegraphics{diagnostics-plot.png}

}

\caption{Diagnostic plot showing residual patterns to assess model
validity}

\end{figure}%

\subsection{Common Pitfalls and
Gotchas}\label{common-pitfalls-and-gotchas}

\begin{tcolorbox}[enhanced jigsaw, opacitybacktitle=0.6, colback=white, toptitle=1mm, breakable, left=2mm, colframe=quarto-callout-warning-color-frame, arc=.35mm, titlerule=0mm, coltitle=black, bottomrule=.15mm, bottomtitle=1mm, title=\textcolor{quarto-callout-warning-color}{\faExclamationTriangle}\hspace{0.5em}{⚠️ Common Mistakes to Avoid}, leftrule=.75mm, opacityback=0, toprule=.15mm, rightrule=.15mm, colbacktitle=quarto-callout-warning-color!10!white]

\begin{enumerate}
\def\labelenumi{\arabic{enumi}.}
\item
  \textbf{Assuming Linearity}: Always visualize your data first!
  Non-linear relationships need different approaches.
\item
  \textbf{Ignoring Outliers}: A few extreme values can drastically
  affect your results. Investigate them carefully.
\item
  \textbf{Extrapolation Dangers}: Don't make predictions far outside
  your observed data range.
\item
  \textbf{Correlation ≠ Causation}: Strong correlations don't prove
  causal relationships.
\item
  \textbf{Sample Size Matters}: Small datasets require extra caution
  with interpretation.
\end{enumerate}

\end{tcolorbox}

\section{Results and Key Findings}\label{results-and-key-findings}

\includegraphics[width=0.5\textwidth,height=\textheight]{../../images/posts/penguins-cinema-4d-4030946_1280.jpg}
\emph{``Presenting our findings!''}

Our analysis revealed several important findings:

\begin{enumerate}
\def\labelenumi{\arabic{enumi}.}
\item
  \textbf{Strong Weight-MPG Relationship}: Vehicle weight explains 75\%
  of variance in fuel efficiency (R² = 0.75), with each additional 1,000
  lbs reducing MPG by \textasciitilde5.3 miles (95\% CI: {[}-6.5,
  -4.1{]})
\item
  \textbf{Cylinder Count Effects}: Cars with fewer cylinders tend to be
  lighter and more fuel-efficient, suggesting cylinder count is
  partially mediated through weight
\item
  \textbf{Model Performance}: The simple linear model provides
  reasonable predictions (RMSE = 3.05 MPG) but shows some systematic
  patterns in residuals, suggesting room for improvement
\item
  \textbf{Practical Implications}: Weight is a strong, reliable
  predictor for quick fuel efficiency estimates
\end{enumerate}

\section{Limitations and
Considerations}\label{limitations-and-considerations}

While this approach is effective, there are important considerations:

\subsection{Model Assumptions}\label{model-assumptions}

\begin{itemize}
\tightlist
\item
  \textbf{Linearity}: The weight-MPG relationship appears reasonably
  linear in the observed range, but may not extend to extreme values
\item
  \textbf{Independence}: Observations are assumed independent, though
  vehicle models may share design characteristics
\item
  \textbf{Homoscedasticity}: Residual variance appears relatively
  constant, though slight heteroscedasticity is visible
\end{itemize}

\subsection{Data Limitations}\label{data-limitations}

\begin{itemize}
\tightlist
\item
  \textbf{Sample Size}: Only 32 observations limits our ability to
  detect subtle effects
\item
  \textbf{Temporal Scope}: Data from 1974 model year; relationships may
  differ for modern vehicles
\item
  \textbf{Vehicle Types}: Limited to passenger cars; findings may not
  generalize to trucks, SUVs, or electric vehicles
\item
  \textbf{Missing Variables}: Many factors affecting fuel efficiency
  (aerodynamics, transmission type, engine technology) are not captured
\end{itemize}

\subsection{Method Limitations}\label{method-limitations}

\begin{itemize}
\tightlist
\item
  \textbf{Simple Model}: Single-predictor model ignores important
  confounding variables
\item
  \textbf{Outlier Sensitivity}: Linear regression can be heavily
  influenced by extreme values
\item
  \textbf{Prediction Range}: Extrapolating beyond observed weight range
  (1.5-5.5 thousand lbs) is risky
\end{itemize}

\section{Practical Applications and
Implications}\label{practical-applications-and-implications}

This analysis has several practical applications:

\textbf{For Data Scientists:} - Template for exploratory regression
analysis - Workflow for model diagnostics and validation - Example of
clear statistical communication

\textbf{For Automotive Analysis:} - Quick fuel efficiency estimation
from weight measurements - Baseline model for evaluating engineering
improvements - Framework for analyzing vehicle characteristics

\textbf{For Learning:} - Hands-on demonstration of regression
assumptions - Practical example of confidence intervals - Template for
reproducible analysis

\section{Future Extensions}\label{future-extensions}

This work could be extended in several directions:

\begin{itemize}
\tightlist
\item
  \textbf{Multiple Regression}: Add cylinder count, horsepower, and
  transmission type
\item
  \textbf{Non-linear Models}: Explore polynomial or spline regression
  for better fit
\item
  \textbf{Interaction Effects}: Test if weight effects differ by
  cylinder count
\item
  \textbf{Modern Data}: Replicate analysis with current vehicle data to
  see how relationships have changed
\item
  \textbf{Causal Analysis}: Use instrumental variables or natural
  experiments to establish causality
\item
  \textbf{Machine Learning}: Compare linear regression to tree-based or
  neural network approaches
\end{itemize}

\section{Conclusion}\label{conclusion}

In this post, we've demonstrated a complete workflow for exploratory
data analysis and simple linear regression. We've seen how vehicle
weight strongly predicts fuel efficiency (R² = 0.75), learned to
validate model assumptions through diagnostics, and discussed important
limitations.

\textbf{Key Takeaways:} - Always start with data exploration before
modeling - Visualize relationships to understand patterns - Validate
assumptions through diagnostic plots - Be honest about limitations and
scope - Connect statistical findings to practical applications

\textbf{Next Steps:} - Try this workflow with your own dataset -
Experiment with multiple predictor variables - Explore the additional
resources below - Share your results and questions in the comments

I encourage you to adapt this approach to your specific use case. The
principles demonstrated here---systematic exploration, rigorous
diagnostics, and honest assessment---apply across domains.

\section{Further Reading and
Resources}\label{further-reading-and-resources}

\subsection{Essential Books}\label{essential-books}

\textbf{For R Programming:} - Wickham, H., \& Grolemund, G. (2017).
\emph{R for Data Science}. O'Reilly Media. https://r4ds.had.co.nz/ -
Free online version covering tidyverse ecosystem - Wickham, H. (2016).
\emph{ggplot2: Elegant Graphics for Data Analysis}. Springer.
https://ggplot2-book.org/

\textbf{For Statistical Modeling:} - James, G., Witten, D., Hastie, T.,
\& Tibshirani, R. (2021). \emph{An Introduction to Statistical Learning
with Applications in R} (2nd ed.). Springer. - Comprehensive, accessible
introduction to modern statistical learning - Fox, J., \& Weisberg, S.
(2019). \emph{An R Companion to Applied Regression} (3rd ed.). Sage. -
Detailed treatment of regression diagnostics and extensions

\subsection{Online Tutorials and
Blogs}\label{online-tutorials-and-blogs}

\textbf{R Programming:} - \href{https://www.r-bloggers.com/}{R-bloggers}
- Aggregated R news and tutorials -
\href{https://www.rstudio.com/blog/}{RStudio Blog} - Official updates
and best practices -
\href{https://towardsdatascience.com/tagged/r-statistics}{Towards Data
Science: R Statistics} - Practical tutorials

\textbf{Statistical Modeling:} -
\href{https://stats.stackexchange.com/}{Cross Validated} - Q\&A for
statistical methodology - \href{https://stats.oarc.ucla.edu/r/}{UCLA
Statistical Consulting} - Excellent R regression tutorials -
\href{https://online.stat.psu.edu/stat501/}{Penn State STAT 501} - Free
online regression course

\subsection{Technical Documentation}\label{technical-documentation}

\textbf{R Packages:} - \href{https://www.tidyverse.org/}{tidyverse
documentation} - Complete reference for tidyverse packages -
\href{https://broom.tidymodels.org/}{broom package} - Tidy model output
- \href{https://ggplot2.tidyverse.org/reference/}{ggplot2 reference} -
Complete plotting functions

\textbf{R Language:} -
\href{https://cran.r-project.org/doc/manuals/r-release/R-lang.html}{R
Language Definition} - Official R documentation -
\href{https://adv-r.hadley.nz/}{Advanced R} - Deep dive into R
programming by Hadley Wickham

\subsection{Academic Papers}\label{academic-papers}

\textbf{Foundational Statistics:} - Box, G. E. P. (1976). ``Science and
Statistics''. \emph{Journal of the American Statistical Association},
71(356), 791-799. - Classic paper on statistical thinking - Gelman, A.,
\& Hill, J. (2006). \emph{Data Analysis Using Regression and
Multilevel/Hierarchical Models}. Cambridge University Press. -
Comprehensive treatment of applied regression

\textbf{Data Visualization:} - Wilkinson, L. (2005). \emph{The Grammar
of Graphics} (2nd ed.). Springer. - Theoretical foundation for ggplot2 -
Tufte, E. R. (2001). \emph{The Visual Display of Quantitative
Information} (2nd ed.). Graphics Press.

\subsection{Community Resources}\label{community-resources}

\textbf{Q\&A and Forums:} -
\href{https://stackoverflow.com/questions/tagged/r}{Stack Overflow R
Tag} - Programming troubleshooting -
\href{https://community.rstudio.com/}{RStudio Community} - Friendly
community support - \href{https://www.reddit.com/r/rstats/}{Reddit
r/rstats} - Discussions and resources

\textbf{Learning Communities:} - \href{https://www.rfordatasci.com/}{R
for Data Science Online Learning Community} - Book club and Slack
workspace - \href{https://rladies.org/}{R-Ladies Global} - Inclusive R
community with worldwide chapters -
\href{https://github.com/rfordatascience/tidytuesday}{TidyTuesday} -
Weekly data project community

\textbf{Conferences and Events:} -
\href{https://user2024.r-project.org/}{useR! Conference} - Annual R user
conference - \href{https://posit.co/conference/}{rstudio::conf} -
RStudio's annual conference -
\href{https://www.meetup.com/topics/r-project-for-statistical-computing/}{Local
R User Groups} - Find meetups near you

\subsection{Data Sources}\label{data-sources}

\textbf{Practice Datasets:} - Built-in R datasets: \texttt{data()} -
Type in R console to see all available datasets -
\href{https://archive.ics.uci.edu/ml/}{UCI Machine Learning Repository}
- Classic benchmark datasets -
\href{https://www.kaggle.com/datasets}{Kaggle Datasets} -
Community-contributed data -
\href{https://github.com/rfordatascience/tidytuesday}{TidyTuesday} -
Weekly practice datasets

\textbf{R Data Packages:} - \texttt{palmerpenguins} - Modern alternative
to iris dataset - \texttt{nycflights13} - Flight data for learning dplyr
- \texttt{gapminder} - International development data

\subsection{Related Topics to Explore}\label{related-topics-to-explore}

\textbf{Next Steps in Your Learning Journey:} - Multiple regression and
variable selection - Generalized linear models (GLM) - Mixed effects
models for hierarchical data - Time series analysis - Machine learning
with tidymodels - Bayesian regression with rstanarm or brms

\begin{center}\rule{0.5\linewidth}{0.5pt}\end{center}

\subsection{Reproducibility
Information}\label{reproducibility-information}

\subsection{Data Availability}\label{data-availability}

\begin{itemize}
\tightlist
\item
  \textbf{Dataset}: mtcars (built-in R dataset)
\item
  \textbf{Access}: Available in all R installations via
  \texttt{data(mtcars)}
\item
  \textbf{Documentation}: \texttt{?mtcars} for variable descriptions
\end{itemize}

\subsection{Code Repository}\label{code-repository}

\begin{itemize}
\tightlist
\item
  \textbf{GitHub}: {[}Link to your repository{]}
\item
  \textbf{Analysis File}: This complete document with all code
\item
  \textbf{License}: {[}Specify license, e.g., MIT, CC-BY-4.0{]}
\end{itemize}

\subsection{Session Information}\label{session-information}

\begin{verbatim}
R version 4.5.1 (2025-06-13)
Platform: aarch64-apple-darwin24.4.0
Running under: macOS Tahoe 26.1

Matrix products: default
BLAS:   /opt/homebrew/Cellar/openblas/0.3.30/lib/libopenblasp-r0.3.30.dylib 
LAPACK: /opt/homebrew/Cellar/r/4.5.1/lib/R/lib/libRlapack.dylib;  LAPACK version 3.12.1

locale:
[1] en_US.UTF-8/en_US.UTF-8/en_US.UTF-8/C/en_US.UTF-8/en_US.UTF-8

time zone: America/Los_Angeles
tzcode source: internal

attached base packages:
[1] stats     graphics  grDevices utils     datasets  methods   base     

other attached packages:
 [1] patchwork_1.3.2 knitr_1.50      broom_1.0.9     lubridate_1.9.4
 [5] forcats_1.0.0   stringr_1.5.1   dplyr_1.1.4     purrr_1.1.0    
 [9] readr_2.1.5     tidyr_1.3.1     tibble_3.3.0    ggplot2_4.0.0  
[13] tidyverse_2.0.0

loaded via a namespace (and not attached):
 [1] utf8_1.2.6         generics_0.1.4     stringi_1.8.7      lattice_0.22-7    
 [5] hms_1.1.3          digest_0.6.37      magrittr_2.0.3     evaluate_1.0.5    
 [9] grid_4.5.1         timechange_0.3.0   RColorBrewer_1.1-3 fastmap_1.2.0     
[13] Matrix_1.7-3       jsonlite_2.0.0     backports_1.5.0    tinytex_0.57      
[17] mgcv_1.9-3         scales_1.4.0       textshaping_1.0.3  cli_3.6.5         
[21] rlang_1.1.6        splines_4.5.1      withr_3.0.2        yaml_2.3.10       
[25] tools_4.5.1        tzdb_0.5.0         vctrs_0.6.5        R6_2.6.1          
[29] lifecycle_1.0.4    ragg_1.4.0         pkgconfig_2.0.3    pillar_1.11.0     
[33] gtable_0.3.6       glue_1.8.0         systemfonts_1.2.3  xfun_0.53         
[37] tidyselect_1.2.1   farver_2.1.2       htmltools_0.5.8.1  nlme_3.1-168      
[41] rmarkdown_2.29     labeling_0.4.3     compiler_4.5.1     S7_0.2.0          
\end{verbatim}

\begin{center}\rule{0.5\linewidth}{0.5pt}\end{center}

\subsection{Share This Post}\label{share-this-post}

Found this helpful? Share it with your network:

\begin{itemize}
\tightlist
\item
  \href{https://twitter.com/intent/tweet?text=Check\%20out\%20this\%20R\%20tutorial&url=YOUR_POST_URL&via=rgt47}{Twitter}
\item
  \href{https://www.linkedin.com/sharing/share-offsite/?url=YOUR_POST_URL}{LinkedIn}
\item
  \href{https://reddit.com/submit?url=YOUR_POST_URL&title=YOUR_POST_TITLE}{Reddit}
\end{itemize}

\subsection{Connect and Discuss}\label{connect-and-discuss}

\emph{Have questions or suggestions? I'd love to hear from you:}

\begin{itemize}
\tightlist
\item
  \textbf{Twitter}: \href{https://twitter.com/rgt47}{@rgt47} - Quick
  questions and discussions
\item
  \textbf{LinkedIn}: \href{https://linkedin.com/in/rgthomaslab}{Ronald
  Glenn Thomas} - Professional networking
\item
  \textbf{GitHub}: \href{https://github.com/rgt47}{rgt47} - Code,
  issues, and contributions
\item
  \textbf{Email}: \href{https://rgtlab.org/contact}{Contact through
  website} - Detailed inquiries
\end{itemize}

\begin{center}\rule{0.5\linewidth}{0.5pt}\end{center}

\subsection{About the Author}\label{about-the-author}

\textbf{Ronald (Ryy) Glenn Thomas} is a biostatistician and data
scientist at UC San Diego, specializing in statistical computing,
machine learning applications in healthcare, and reproducible research
methods. He develops R packages and conducts research at the
intersection of statistics, data science, and clinical research.

\emph{Connect: \href{https://rgtlab.org}{Website} \textbar{}
\href{https://orcid.org/0000-0003-1686-4965}{ORCID} \textbar{}
\href{https://scholar.google.com/citations?user=YOUR_ID}{Google
Scholar}}

\begin{center}\rule{0.5\linewidth}{0.5pt}\end{center}



\end{document}
